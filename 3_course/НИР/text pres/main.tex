\documentclass[12pt,a4paper]{article}
\linespread{1,15}
\usepackage[utf8]{inputenc}
\usepackage[T2A]{fontenc}
\usepackage[russian]{babel}
\usepackage{amsmath}
\usepackage{amsfonts}
\usepackage{amssymb}
\usepackage{graphicx}
\usepackage{listings}
\usepackage[title,titletoc]{appendix}
\usepackage[left=30mm, top=20mm, right=15mm, bottom=20mm, nohead, footskip=10mm]{geometry}
\usepackage{indentfirst}
\author{Александр Лаптев}
\title{Создание информационной системы для организации работы предприятия}

\begin{document}

\section{Актуальность}

На сегодняшний день практически у каждого есть какое-либо портативное устройство или другая потребительская электроника. В такой технике приоритетом является низкое энергопотребление в сочетании с относительным быстродействием. Этим требованиям отвечают семейства микроконтроллеров AVR и ARM.

Встраиваемые системы и портативные устройства --- одна из самых быстрорастущих технологических сфер. И, поэтому, разработка под микроконтроллеры и микропроцессоры этих семейств является актуальной.

\section{Цель и задачи}

Целью работы: разработка программного продукта под микроконтроллер одного из рассмотренных семейств (AVR, ARM), с использованием выбранной программной и аппаратной платформ.
    
Задачи работы:

\begin{enumerate}
    \item Рассмотрение имеющихся аппаратных и программных средств для выбора наиболее подходящей платформы для разработки под AVR и ARM микроконтроллеры.
    
    \item Разработка собственного программного продукта на выбранной аппаратно-программной платформе.
\end{enumerate}

\section{Правила генерации}

Генерация персонажа представляет собой совокупность из параметров персонажа, которые пользователь может выбрать из предложенных вариантов, а также характеристик, которые одинаковы для всех персонажей, но значения которых генерируются случайным образом.

Изначально, пользователю предоставляется возможность выбрать то, кем он, непосредственно, хочет быть в игре (раса и класс персонажа). После чего происходит генерация значений для характеристик персонажа.

Для определения значений характеристик персонажа существует несколько различных способов. В данном случае, характеристики персонажа генерируются следующим образом:
\begin{enumerate}
    \item для каждой из характеристик производится четыре броска игральной кости;
    
    \item меньшее из выброшенных значений исключается из генерации;
    
    \item оставшиеся три значения суммируются. Полученное значение и будет являться значением для характеристики.
\end{enumerate}

Для некоторых других характеристик персонажа, таких как хит\-поинты и класс защиты, генерация осуществляется иначе.

Для расчета хит-поинтов для героя первого уровня берется максимальное значение кости хитов, которое определяется классом выбранного персонажа и модификатор телосложения, который рассчитывается исходя из соответствующей характеристики. Полученные значения складываются, что и дает количество хит-поинтов персонажа.

Для расчета класса защиты для героя первого уровня используется значение 10 (базовое значение для персонажа, который не носит броню) и модификатор ловкости, рассчитанный исходя из характеристики ловкости. Данные значения складываются, выводя значение, которое и определяет класс защиты персонажа.

Модификаторы телосложения, ловкости и других характеристик рассчитываются одинаково. Формула для расчета модификатора конкретной характеристики выглядит следующим образом: $MC = (CH - 10) / 2$, где MC --- модификатор характеристики, CH --- конкретное значение характеристики, для которого производится расчет. Полученный результат округляется в меньшую сторону. Модификатор характеристики может принимать значение от $-5$ до $+10$.

\section{mainText}

Эта функция представляет собой реализацию приветственного окна, в котором сообщается о назначении программы (генератор персонажа DnD) и указывается, что для продолжения работы нужно нажать соответствующую кнопку.

Помимо этого, после нажатия кнопки внутри функции генерируются псевдослучайные числа, которые имитируют броски игральной кости. Генерируется четыре псевдослучайных числа и находится сумма этих чисел, исключая меньшее. Такие манипуляции производятся шесть раз (по количеству характеристик персонажа).

\section{genRacePers}

В этой функции реализована часть генератора, которая отвечает за выбор пользователем расы персонажа. На экран выводятся двенадцать вариантов рас, среди которых пользователь может переключаться нажатием на кнопки: <<ВВЕРХ>> (<<UP>>), <<ВНИЗ>> (<<DOWN>>), <<ВПРАВО>> (<<RIGHT>>), <<ВЛЕВО>> (<<LEFT>>); чтобы подтвердить свой выбор требуется нажать кнопку <<ВЫБРАТЬ>> (<<SELECT>>).

\section{genClassPers}

В этой функции реализована возможность выбора класса персонажа. На экран последовательно выводятся тринадцать предусмотренных классов, между которыми также можно переключаться с помощью кнопок, для подтверждения выбора предусмотрена кнопка <<ВЫБРАТЬ>> (<<SELECT>>).

\section{charactPers}

С помощью данной функции на экран выводятся значения характеристик, сгенерированные в функции mainText, а также на их основе высчитываются количество хит-поинтов и класс защиты для выбранных расы и класса, которые также выводятся на экран. Навигация здесь, также как и в предыдущих функциях осуществляется с помощью кнопок.

\section{clickButton}

Эта функция необходима для того, чтобы обрабатывать нажатия кнопок, она возвращает целое число, которое будет указывать на то, нажата ли кнопка и, если нажата, то какая.

\section{setup, loop}

Помимо этих функций в проектах, написанных в Arduino IDE есть еще две стандартных функции. В функции setup задается скорость передачи данных для последовательного порта и инициализируется lcd-дисплей. В функции loop вызывается функция mainText для того, чтобы генератор персонажа работал вплоть до отключения его от питания.

\section{Недостатки}

Встроенные функции в Arduino IDE зачастую имеют в своем составе ряд дополнительных проверок, чтобы минимизировать количество возможных ошибок, которые могут появиться при компиляции.
    
Также, довольно большое количество глобальных переменных, а также их не оптимальная типизация, из-за которой они могут занимать в памяти больше места, чем им требуется в действительности, также увеличивают количество потребляемых ресурсов микроконтроллера.

Помиом этого, на работу программы негативное влияние оказывает использование встроенной функции задержки delay, поскольку, при использовании этой функции, приостанавливается работа всей программы.

Поскольку в данном генераторе реализованы лишь базовые возможности, то в будущем есть возможность для добавления в генератор ряда функций, например, выбор снаряжения на основе выбранного класса или расчет бросков атаки и урона персонажа, чтобы сделать генератор пригодным для полноценного создания персонажа и облегчения расчетов во время игры.

Кроме того, отладочная плата имеет довольно большие габариты и часть периферии не используется, в будущем планируется развести и вытравить собственную плату для уменьшения габаритов устройства, также возможна замена дисплея на больший по размерам, поскольку количество строк и символов в строке дисплея на плате расширения не позволяет пользователю комфортно работать с ним.

\section{Заключение}

В ходе выполнения научно-исследовательской работы было проведено знакомство с различными платформами для разработки под микроконтроллеры, проведен обзор различных семейств микроконтроллеров и выбрана платформа для разработки собственного программного продукта --- генератора персонажа DnD.
    
Был реализован основной функционал:

\begin{enumerate}
    \item наличие возможности выбора расы и класса персонажа;
    
    \item генерация случайных значений для характеристик персонажа, согласно правилам DnD;
    
    \item расчет некоторых дополнительных характеристик персонажа с использованием сгенерированных для основных характеристик значений.
\end{enumerate}

В результате, был создан вариант генератора персонажа DnD.

\end{document}
