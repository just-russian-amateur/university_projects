\documentclass[12pt,a4paper]{article}
\linespread{1}
\usepackage[utf8]{inputenc}
\usepackage[T2A]{fontenc}
\usepackage[russian]{babel}
\usepackage{amsmath}
\usepackage{amsfonts}
\usepackage{amssymb}
\usepackage{graphicx}
\usepackage{listings}
\usepackage[title,titletoc]{appendix}
\usepackage[left=15mm, top=20mm, right=15mm, bottom=20mm, nohead]{geometry}
\usepackage{indentfirst}

\title{Property Theme FQW Text}
\author{laptevaleksandr2001 }
\date{October 2022}

\begin{document}

\subsection{Проектирование и разработка устройства для генерации персонажа DnD}

Целью ВКР является проектирование и создание устройства, которое будет предназначено для генерации персонажа DnD, а также разработать программное обеспечение для данного устройства, чтобы оно имело весь функционал, необходимый для помощи игрокам на протяжении всей игры, как минимум не уступающий Web-аналогам.

\subsection{Задачи}

Среди задач, которые должны быть решены в данной работе можно выделить следующие:
\begin{enumerate}
    \item Аналитическое сравнение возможностей различных микроконтроллеров, с целью подобрать оптимальный вариант по соотношению "цена-качество";

    \item Разработка принципиальной схемы устройства и подбор компонентов, необходимых для его сборки;

    \item Проектирование и трассировка печатной платы, установка на нее всех компонентов схемы;

    \item Сборка и корпусирование готового устройства;

    \item Разработка программного обеспечения для помощи игрокам на любом этапе игры, согласно правилам DnD5;

    \item Отладка и тестирование работоспособности программной и аппаратной части устройства.
\end{enumerate}

\subsection{Актуальность}

На сегодняшний день DnD является, пожалуй, самой популярной стратегией, которая собрала вокруг себя огромное сообщество игроков.

Для DnD разработано огромное количество Web-ресурсов и ботов от энтузиастов и крупных компаний, которые облегчают игрокам процесс игры и генерации своего персонажа. В то же время, это позволяет игрокам играть не честно, либо подсматривать в правила. Чтобы избежать подобных манипуляций во время игры, предлагается разработать устройство, которое выполняло бы те же самые функции, но было бы портативным, чтобы его можно было носить с собой и не имело выхода в интернет, чтобы исключить обман и добавить в игровой процесс чуть больше самобытности.

О масштабах распространения игры можно судить по популярности онлайн сервиса DnD Beyond. Этот сервис, на сегодняшний день является по сути эталоном среди всех вспомогательных приложений для игры в DnD. Относительно недавно он был куплен компанией Hasbro (принадлежат права на DnD) за 146.3 млн долларов. Также, о популярности сервиса, да и в целом игры говорит и количество пользователей -- их насчитывается около 10 млн.

\subsection{Промежуточный и планируемый результаты работы}

На данный момент, есть работающий вариант генератора персонажа, собранный на отладочной плате Arduino Uno, с использованием платы расширения с LCD-дисплеем, а также, реализован базовый функционал приложения для генерации персонажа с последовательным меню и генерацией параметров, которые необходимы в начале игры.

В ходе выполнения работы планируется выполнить все поставленные задачи и в результате предоставить полноценное работоспособное устройство, помещенное в корпус, а также продемонстрировать работоспособность разработанного программного обеспечения на этом устройстве.

\end{document}
