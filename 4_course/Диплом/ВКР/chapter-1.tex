\chapter{Создание персонжа для Dangeons \& Dragons, основные принципы и трудности при создании персонажа}

Dungeons \& Dragons --- одна из самых популярных настольных ролевых игр в мире. Это совместная игра с рассказыванием историй, в которой игроки берут на себя роли разных персонажей в истории. Во время игры персонаж будет заводить друзей и врагов, сражаться с монстрами, находить добычу и выполнять квесты~\cite{explorednd,DandD,wiki}.

Основной целью игры является описание персонажа и его приключений. Для создания персонажа могут быть задействованы игральные кости и Основные правила, а также личные предпочтения игроков.

Ключевыми действующими лицами в игре являются Мастер Подземелий и несколько игроков. Мастер нужен для управления ходом игры и повествованием истории. Игроки являются основными действующими лицами в повествовании и берут на себя роль того персонажа, которого они создали для себя.

В задачи, которые выполняет Мастер входят следующие:

\begin{enumerate}
    \item Описание окружающей местности, локаций, в которых находятся персонажи.
    \item Повествует о ходе событий, действующими лицами которых являются персонажи.
    \item Описание препятствий или головоломок, которые могут попадаться на пути игроков.
    \item Обычно Мастер также играет от лица второстепенных неигровых персонажей.
\end{enumerate}

Основным средством для описания персонажа для игрока служит так называемый лист персонажа. Лист персонажа представляет собой сгруппированный в удобном виде набор характеристик персонажа. Характеристики описывают все то, что данный персонаж может делать, взаимодействуя с окружающей средой. К таким характеристикам относятся следующие:

\begin{enumerate}
    \item Основные атрибуты, присущие каждому персонажу: Сила, Телосложение, Ловкость, Интеллект, Мудрость, Харизма.
    \item Особые умения, которые индивидуализируют персонажа: способность к убеждению или расследованию.
    \item Определенные действия, такие как атака оружием или произнесение заклинаний.
    \item Языки, на которых говорит персонаж, или инструменты, которыми он умеет пользоваться.
\end{enumerate}

Когда мастером были описаны задачи, которые поставлены перед игроками, игроки могут приступить к описанию своих действий. После чего Мастер определяет то, какие последствия действий игрока~\cite{DnD}.

В ходе игры игроки могут бросать кости, чтобы увидеть, добьются ли они успеха, или могут ли они рассказать о действии, или продолжится ли история. Бросок игрока может изменяться в зависимости от навыков персонажа на листе. Мастер также в некоторых случаях бросает кости, если это требуется в ходе повествования.

Традиционно руководство игры или свод правил включает в себя три книги: «Player’s Handbook», «Dungeon Master’s Guide» и «Monster Manual». Кроме этого существует множество различных дополнений, которые Мастер может использовать или не использовать~\cite{blackcitadelrpg}.

Иногда для игры используются различные карты, чтобы визуально изобразить ситуации в игре, также могут использоваться фигурки персонажей и их противников. Но основным, а иногда и единственным требованием для игры является наличие листов персонажей и набора игральных костей с заданным числом сторон (d4, d6, d8, d10, d12, d20, d100 (процентовый кубик)).

Создание персонажа с нуля --- долгий и кропотливый процесс. Ниже описаны основные шаги для создания персонажа.

Перед началом игры стоит заранее определиться с тем, будет ли персонаж сбалансированным и гармоничным в плане характеристик и описания или же более случайным. От этого выбора может зависеть то, по каким принципам будут генерироваться те или иные характеристики персонажа.

Начать можно с выбора расы и класса своего персонажа или же заранее сгенерировать характеристики для персонажа и распределить их позже для того, чтобы получить более сбалансированного героя.

\section{Определение базовых характеристик персонажа}

Многое что делает персонаж в игре зависит от его характеристик, всего их шесть: Сила, Телосложение, Ловкость, Интеллект, Мудрость и Харизма. Каждая характеристика представляет из себя какое-то количество очков --- число, которое будет записано в листе персонажа~\cite{DnDWiki, Basicrules}.

В самом начале генерирация очков характеристик осуществляется в случайном порядке. 
Классически генерация характеристик происходит броском 3d6 (трёхкратным броском шестигранного кубика), что даёт разброс от 3 до 18, но обычно используется бросок 4d6, причём значение кости, на которой выпало наименьшее число, отбрасывается, а значения остальных складываются. Такой вариант также даёт распределение от 3 до 18, но более смещённое в сторону 18: вероятность появления минимума снижается в 6 раз, а вероятность появления максимума увеличивается в 2 раза. В 5 редакции основным методом определения характеристик является выбор из набора стандартных значений [15, 14, 13, 12, 10, 8]. Кроме этих двух, есть и другие альтернативные способы, описание которых дано в Книге игрока (Для версии 3.5 в книге Мастера (Dungeon Master’s Guide)). Описанные действия необходимо повторить 6 раз, полученные значения и будут значениями характеристик персонажа~\cite{DandDforms, longstoryshort}.

Числа, которые были определены в результате бросков -- это только основа характеристик персонажа, они не отображают полной картины. В процессе игры персонаж может изменить свои параметры, улучшить их благодаря полученному опыту.

Еще один метод для генерации, который иногда используется заключается в следующем: персонаж начинает со всеми характеристикам равными 8. У игрока есть 27 очков, которые он может свободно распределить, при этом 14 и 15 значение характеристики стоят по 2 очка каждое.

В некоторых случаях Мастер может предложить игрокам иную систему расчета характеристик, но здесь были рассмотрены наиболее распространенные способы генерации.

\section{Выбор расы персонажа}

Каждый персонаж принадлежит расе --- виду, в мире фэнтези. Самые общие игровые расы --- это дварфы, эльфы, халфлинги и люди. Существует еще много иных рас, которые могут быть доступны на усмотрение Мастера~\cite{screenrant, DandDforms}.

Выбранная раса вносит важный вклад в становление персонажа, она определяет его внешний вид, природные таланты и родословную. Раса персонажа, в частности определяет его расовые черты, такие как корректировка очков характеристик, особые чувства, талант к использованию определенного оружия, или способность использовать малые заклинания. Иногда эти черты соответствуют возможностям определенных классов~\cite{PlayersHandbook, MonsterManual, MastersGuide}.

Расовые черты персонажа также заносятся в лист персонажа.

\section{Выбор класса персонажа}

Каждый персонаж --- член класса. Класс хорошо описывает профессию, которой следует персонаж, какими специальными талантами он обладает, и тактика которую он скорее всего будет использовать во время прохождения подземелий, боя с монстрами и ведения напряженных переговоров~\cite{VolosGuidetoMonsters}.

Самые распространенные классы --- это жрец, боец, разбойник и маг. Жрецы это священнослужители своих богов, наделенные толикой их силы, бойцы это жестокие воины и специалисты по оружию, разбойники это специалисты во многих областях знания и надувательства, маги это мастера тайной магии. Другие классы могут быть доступны на усмотрение Мастера.

Персонаж также получает некоторые преимущества от выбранного класса. Многие из этих преимуществ это особенности класса, отличающие его от остальных~\cite{MORDENKAINENSTOMEOFFOES, XanatharsGuidetoEverything, SwordCoastAdventurersGuide}.

Вся базовая информация и особенности выбранного класса записываются в лист персонажа.

\section{Выбор предыстории персонажа}

У каждого персонажа есть предыстория, описывающая откуда он родом, чем он занимался, и местонахождение персонажа в мире D\&D.

Игрок может выбрать предысторию данную в описании класса персонажа, или выбрать историю из раздела <<Предыстории и навыки>>. Мастер может предлагать дополнительные предыстории, помимо описанных в указанном разделе.

Предыстория предоставляет персонажу новые особенности, такие как обученность конкретным навыкам. Полученные особенности и изученные навыки также записываются в лист персонажа.

\section{Выбор специализации}

Класс говорит о том, что персонаж будете делать. А специализация о том, как персонаж будет это делать. Специализация определяет предпочтительную тактику боя персонажа, и его методы расследования --- качества, которые могут возникать при сосредоточенном обучении или дароваться талантами. Специализации не обязательны, и Мастер может не использовать их.

Игрок может выбрать специализацию предложенную в описании класса персонажа, или выбрать иную из раздела <<Специализации и таланты>>. Мастер может предлагать дополнительные специализации, помимо описанных в указанном разделе.

Специализации дает особенные способности, называемые талантами, и предоставляет дополнительные по мере роста вашего уровня.

Талант также записывается в лист персонажа.

\section{Распределение очков характеристик персонажа}

Теперь, когда выбрана раса и класс персонажа, можно будет выбрать, как же лучше всего распределить очки характеристик. Например, если выбран боец, то, возможно, сделать наивысшей характеристикой его силу, или, если выбран высший эльф как раса, то будет получен бонус к интеллекту, что хорошо подходит к классу мага.

Очки, сгенерированные случайным образом в самом начале, нужно распределить между шестью описанными ранее базовыми характеристиками. После того, как данные изменения будут проведены, также следует скорректировать очки в характеристиках, которые зависят от расы и класса персонажа, при этом суммарное значение характеристики не должно быть выше 20 очков.

Это хорошее время для определения модификаторов характеристик. Запись модификаторов осуществляется напротив значения ваших характеристик.

\section{Выбор снаряжения}

Предыстория персонажа и его класс, предлагают набор начальной экипировки, включающий оружие, броню и другие приключенческие принадлежности. Игрок может выбрать эти наборы чтобы быстрее начать игру.

Кроме того, игра предоставляет возможность купить дополнительное снаряжение. Мастер может определить количество доступных игрокам денежных средств.

Основополагающей частью игры, как уже писалось ранее, являются приключения персонажа. По ходу игры персонажу могут попадаться различные монстры и другие второстепенные неигровые персонажи, персонаж может иметь дело с различными ловушками и другими опасностями в окружающих их условиях. Поэтому, в игре предусмотрено несколько дополнительных очень важных характеристик, которые определяют то, насколько вынослив персонаж и как он переносит сражения и прочие опасности. Это следующие характеристики:  хит-поинты, класс защиты, модификатор инициативы и бонусы к атакам.

Хит-поинты персонажа определяют насколько сложно ему приходится в бою и в некоторых других опасных ситуациях. Описание класса персонажа указывает как рассчитать это число, которое так же является максимальным значением хит-поинтов персонажа. С ростом уровня, максимальное число хит-поинтов так же увеличивается. Количество хит-поинтов, как и все остальные характеристики, записываются на листе персонажа.

Модификатор ловкости персонажа, броня и щиты, размер, а также другие особенности принимают участие в расчете класса защиты --- показателя того, насколько хорошо персонаж уходит от атак. Если персонаж не носит броню, то класс защиты равен 10 + модификатор ловкости персонажа. В противном случае, используются числа, предоставленные в разделе <<Экипировка>> для брони и щитов для расчета класса защиты.

Персонаж вступает в бой в последовательности определенной его инициативой. Модификатор инициативы персонажа равен модификатору его ловкости плюс любые классовые, расовые или иные особенности.

Персонаж может совершать два вида атак: ближнюю и дальнюю. Модификатор для атак ближнего боя равен модификатору силы или ловкости плюс бонусы или штрафы от других источников. Модификатор для атак дальнего боя равен модификатору ловкости плюс бонусы или штрафы от других источников.

Некоторые персонажи могут использовать заклинания. Если персонаж один из таких, то в описании класса будет указана основная характеристика (обычно это интеллект или мудрость), используемая членами этого класса для сотворения заклинаний. Механика расчета модификатора атаки, в таком случае аналогична правилам для ближнего и дальнего боя, за исключением использования модификаторов для иных характеристик.

Против некоторых заклинаний можно сделать спасбросок. Описание класса персонажа объясняет как рассчитать уровень сложности для таких спасбросков, направленных на избежания эффекта от ваших заклинаний. Запишите этот уровень сложности в листе вашего персонажа.

Класс персонажа, помимо всего прочего, может предоставлять бонус к броскам атаки с использование оружия или заклинаний. Если таблица в описании класса персонажа включает колонку <<Атака оружием>>, следует добавить число для персонажа 1-го уровня к модификатору атаки оружием; если таблица включает колонку <<Атака заклинанием>>, то следует добавить число для персонажа 1-го уровня к модификатору атаки заклинаниями персонажа.

После всех проведенных манипуляций игрок может описать физические и личностные качества персонажа. Кроме того, игроку предстоит выбрать подходящее имя для персонажа. Описание расы вашего персонажа предоставляет примеры имен для членов этой расы.

Также в рамках предыстории игрок может описать внешний вид персонажа, его цели, мотивацию, мировоззрение и в целом его личность.

\section{Начисление очков опыта}

По мере того как персонаж проходит приключения и преодолевает испытания, он получает очки опыта. Персонаж набравший определенное количество очков опыта переходит на следующий уровень.

При получении нового уровня, класс персонажа может предоставить ему дополнительную способность, данную в описании класса. Персонаж может получить новые черты. Кроме того, каждый четвертый уровень дает персонажу два дополнительных очка, с помощью которых он может увеличить одну из своих характеристик; в соответствии с правилами, любая характеристика персонажа не может превышать значение 20.

Следующая таблица отображает прогресс персонажа от уровня к уровню, без учета особенностей классов. В таблице указано как много очков опыта необходимо для перехода на следующий уровень.

\begin{table}[H]
    \centering
    \caption{Прогресс персонажа для каждого уровня}
    \begin{tabular}{|l|l|l|}
        \hline
        Опыт & Уровень & Бонусы \\
        \hline
        0 & 1 & Предыстория, специализации, способности \\
        \hline
        250 & 2 & - \\
        \hline
        950 & 3 & Способность (необязательно) \\
        \hline
        2250 & 4 & +1 к двум характеристикам \\
        \hline
        4750 & 5 & - \\
        \hline
        9500 & 6 & Способность (необязательно) \\
        \hline
        16000 & 7 & Улучшение навыков \\
        \hline
        25000 & 8 & +1 к двум характеристикам \\
        \hline
        38000 & 9 & Способность (необязательно) \\
        \hline
        56000 & 10 & - \\
        \hline
        77000 & 11 & - \\
        \hline
        96000 & 12 & +1 к двум характеристикам, Улучшение навыков \\
        \hline
        120000 & 13 & - \\
        \hline
        150000 & 14 & - \\
        \hline
        190000 & 15 & - \\
        \hline
        230000 & 16 & +1 к двум характеристикам \\
        \hline
        280000 & 17 & Улучшение навыков \\
        \hline
        330000 & 18 & - \\
        \hline
        390000 & 19 & - \\
        \hline
        460000 & 20 & +1 к двум характеристикам \\
        \hline
    \end{tabular}
    \label{tab:my_label}
\end{table}

Из-за наличия такого продуманного персонажа игра считается очень трудоемкой и сложной и требует от игроков большого количества расчетов как на стадии подготовки к игре, так и в ходе самой игры. Это особенно трудно для новых игроков, которые лишь постигают правила игры, так и для Мастеров, которым часто необходимо создать множество неигровых персонажей для более разнообразной и интересной игры. Для более опытных игроков такое устройство позволяет упростить создание персонажа, поскольку разбивает его создание на отдельные шаги, по которым удобно проходить, выбирая те или иные характеристики последовательно и избавляя игрока от физических бросков костей и ручного расчета итоговых характеристик.

Одним из самых трудоемких процессов в игре является подготовительный, когда игрок только создает своего персонажа и генерирует его начальные характеристики и снаряжение. Для облегчения этого процесса создано много онлайн-генераторов и даже мобильных приложений, которые с разной степенью охвата правил выполняют такую генерацию.