\documentclass{altsu-bachelor}
\linespread{1,5}
\title{Проектирование и разработка устройства для генерации персонажа Dungeons \& Dragons}
\author{А.\,В.~Лаптев}
\groupnumber{595}
\GradebookNumber{152}
\supervisor{И.\,А.~Шмаков}
\supervisordegree{ст. преп.}
\ministry{Министерство науки и высшего образования}
\country{Российской Федерации}
\fulluniversityname{ФГБОУ ВО Алтайский государственный университет}
\institute{Институт цифровых технологий, электроники и физики}
\department{Кафедра вычислительной техники и электроники}
\departmentchief{В.\,В.~Пашнев}
\departmentchiefdegree{к.ф.-м.н., доцент}
\shortdepartment{ВТиЭ}
\ChairmanOfTheStateCertificationCommission{С.\,П.~Пронин}
\ChairmanOfTheStateCertificationCommissiondegree{д.т.н., проф.}
\NormController{А.\,В.~Калачёв}
\NormControllerdegree{к.ф.-м.н., доцент}
\Consultant{}
\Consultantdegree{}
\UDC{004.3,004.4}
\docname{БР 09.03.01}

\abstractRU{Данная работа посвящена разработке устройства для генерации персонажа Dungeons \& Dragons. Результатом работы является устройство, которое облегчает игрокам генерацию их персонажа для игры и выполняет большую часть игровых расчетов.}
\abstractEN{Большой текст на английском!}
\keysRU{Arduino, микроконтроллер, генератор персонажа D\&D}
\keysEN{computer simulation, distributed version control}

\date{\the\year}

% Подключение файлов с библиотекой.
\addbibresource{graduate-students.bib}

\begin{document}

\maketitle

\setcounter{page}{2}
\makeabstract
\tableofcontents

\chapter*{Введение}
\addcontentsline{toc}{chapter}{Введение}

На сегодняшний день Dungeons \& Dragons является, пожалуй, самой популярной настольной ролевой игрой, которая собрала вокруг себя огромное сообщество игроков.

Для игры в эту настольную ролевую игру каждый игрок должен создать своего персонажа с определенными навыками и характеристиками. Создание персонажа довольно длительный и кропотливый процесс, требующий знания правил игры, которые довольно обширны. Данное устройство нужно для того, чтобы облегчить игрокам, в особенности начинающим, генерацию их игрового персонажа и выполнять большинство расчетов игровых характеристик вместо них. Также, наличие такого устройства у игрока позволит ему сохранить своего персонажа или нескольких персонажей при долгой игре и защитить его от подмены или утраты.

Для D\&D разработано огромное количество Web-ресурсов и ботов от энтузиастов и крупных компаний, которые облегчают игрокам процесс игры и генерации своего персонажа.

О масштабах распространения игры можно судить по популярности онлайн сервиса D\&D Beyond. Этот сервис, на сегодняшний день является некоторым эталоном среди всех вспомогательных приложений для игры в D\&D. Относительно недавно он был куплен компанией Hasbro (принадлежат права на D\&D) за 146.3 млн долларов. Также, о популярности сервиса и в целом игры говорит и количество пользователей --- их насчитывается около 10 млн.

Отличие моей работы от существующих аналогов в том, что до сих пор не существует портативного устройства, которое предназначено специально для этих целей и которое человек мог бы взять с собой и сгенерировать персонажа в любом месте, даже без подключения к Интернету. Все решения, известные до сих пор требуют наличие мобильного телефона или другого устройства, имеющего выход в Интернет.

Цель работы: спроектировать и разработать портативное устройство для генерации персонажа D\&D, которое будет приспособлено для повседневного использования в любом месте без подключения к Интернету.

Задачи, которые необходимо решить для достижения цели:

\begin{enumerate}
    \item Подбор компонентов, необходимых для прототипирования.
    \item Проектирование функциональных возможностей приложения.
    \item Реализация в программном коде алгоритмов согласно базовым правилам D\&D5.
    \item Отладка и тестирование работоспособности программной и аппаратной части прототипа устройства.
    \item Сборка прототипа устройства.
\end{enumerate}

% Теоретическая глава про все виды CAPTCHA

\chapter{Современные методы защиты от ботов и спама на основе CAPTCHA}

\section{История CAPTCHA}

Проверочный код CAPTCHA -- это метод защиты, основанный на принципе 
аутентификации «вызов-ответ». Он предназначен для предотвращения автоматических 
действий, таких как спам или попытки взлома учетных записей, путем выполнения 
пользователем простого теста, подтверждающего, что он человек, а не программа 
[1].

CAPTCHA является важной мерой безопасности, так как предотвращает автоматические 
атаки, например, массовую регистрацию ботов, и защищает данные пользователя. 
Современные системы CAPTCHA используют не только текст, но и изображения, аудио, 
поведенческие анализы и другие инновационные подходы, чтобы сделать тесты удобными 
для людей, но сложными для программ.

На сегодняшний день наиболее распространенные виды CAPTCHA включают:

\begin{enumerate}
    \item reCAPTCHA -- разработанная Google система, которая предлагает тесты 
    на основе распознавания объектов, анализа поведения или текстовых символов.
    \item hCAPTCHA -- альтернатива reCAPTCHA, фокусирующаяся на защите 
    конфиденциальности пользователей.
    \item Capy -- система CAPTCHA, предлагающая пользователю головоломки, 
    например, сборку изображения или взаимодействие с элементами интерфейса [2].
\end{enumerate}

\section{reCAPTCHA}

reCAPTCHA -- система защиты от автоматизированных действий, разработанная Google, 
которая помогает различать человека и бота. Она объединяет несколько подходов, 
делая проверку удобной для пользователей, но сложной для автоматических систем 
[3].

reCAPTCHA включает в себя следующие версии:

\begin{enumerate}
    \item reCAPTCHA v1 (устарела в 2018 году):
    \begin{enumerate}
        \item пользователи вводили текст, состоящий из искаженных слов, 
        отображаемых на изображении;
        \item использовала слова из книг и документов, которые не могли быть 
        распознаны OCR.
    \end{enumerate}
    \item reCAPTCHA v2:
    \begin{enumerate}
        \item клик по флажку: пользователи подтверждают, что они не роботы, 
        нажимая на флажок «Я не робот»;
        \item выбор объектов на изображениях: пользователи идентифицируют 
        заданные объекты на сетке из картинок;
        \item аудио CAPTCHA: для пользователей с ограничениями зрения, 
        предлагается прослушать запись и ввести услышанные символы.
    \end{enumerate}
    \item reCAPTCHA v3:
    \begin{enumerate}
        \item полностью работает в фоновом режиме, анализируя поведение 
        пользователя на странице;
        \item не требует явных действий, если пользователь считается 
        низкорискованным [4].
    \end{enumerate}
\end{enumerate}

\section{hCAPTCHA}

hCAPTCHA -- это альтернативная система CAPTCHA, разработанная для защиты сайтов 
от ботов и спама, при этом уделяющая особое внимание конфиденциальности 
пользователей. Она стала популярной благодаря своей гибкости и ориентации на 
защиту данных [5].  

Основные особенности hCAPTCHA:

\begin{enumerate}
    \item конфиденциальность:
    \begin{enumerate}
        \item в отличие от reCAPTCHA, hCAPTCHA не собирает данные о 
        пользователях для рекламных целей, что делает ее привлекательной с точки 
        зрения соблюдения конфиденциальности.
    \end{enumerate}
    \item простота интеграции:
    \begin{enumerate}
        \item легко интегрируется с web-сайтами через API;
        \item совместима с большинством популярных платформ, таких как WordPress, 
        и может быть настроена для разных типов взаимодействия.
    \end{enumerate}
    \item модели монетизации:
    \begin{enumerate}
        \item владельцы сайтов могут зарабатывать, разрешая hCAPTCHA 
        использовать проверочные задачи, связанные с машинным обучением, 
        например, разметку данных.
    \end{enumerate}
\end{enumerate}

Виды взаимодействия с пользователями:

\begin{enumerate}
    \item графическая CAPTCHA: выбор изображений, соответствующих запросу;
    \item текстовая CAPTCHA: ввод символов (редко используется);
    \item аудио CAPTCHA: для пользователей с ограниченными возможностями, 
    предлагается прослушать и ввести услышанные символы;
    \item клик CAPTCHA: нажатие на флажок «Я не робот» (для низкорискованных 
    пользователей).
\end{enumerate}

\section{Capy}

Capy CAPTCHA -- это инновационная система CAPTCHA, разработанная с акцентом на 
удобство для пользователей и адаптацию к современным web-средам. Она предлагает 
интерактивные методы проверки, направленные на минимизацию раздражения 
пользователей при сохранении высокого уровня защиты от ботов [6].

Основные особенности Capy CAPTCHA:

\begin{enumerate}
    \item интерактивность:
    \begin{enumerate}
        \item Capy использует методы проверки, которые требуют не просто ввода 
        текста или выбора картинок, а выполнения задач, таких как перемещение 
        объектов;
        \item простые задачи делают процесс проверки менее раздражающим и более 
        интуитивным;
    \end{enumerate}
    \item гибкость настройки:
    \begin{enumerate}
        \item система может быть адаптирована под конкретные нужды сайта, 
        включая выбор сложности задач и дизайн интерфейса.
    \end{enumerate}
    \item доступность:
    \begin{enumerate}
        \item подходит для пользователей с различными потребностями, включая 
        мобильные устройства.
    \end{enumerate}
\end{enumerate}

Виды взаимодействия с пользователями:

\begin{enumerate}
    \item головоломки (Puzzle CAPTCHA): сборка пазла с перемещением недостающих 
    элементов в нужное место;
    \item тесты на логику и распознавание: выбор нужного объекта или 
    логического варианта из предложенных;
    \item текстовая CAPTCHA (редко используется).
\end{enumerate}

Capy CAPTCHA используется на сайтах, где важны как защита от ботов, так и 
положительный пользовательский опыт. Особенно популярна в проектах с высоким 
акцентом на дизайн и пользовательское взаимодействие.


% Глава про разработку всех решателей

\chapter{Методология решения задач CAPTCHA}

\section{Общие подходы к автоматизированному решению CAPTCHA}

Для автоматизации решения CAPTCHA могут применяться различные методы и подходы, 
которые зависят от конкретной реализации CAPTCHA. В рамках данной работы можно 
выделить следующий список таких методов:

\begin{enumerate}
    \item cинтез речи;
    \item сегментация;
    \item классификация;
    \item распознавание последовательности;
    \item может быть продлю...
\end{enumerate}

% TODO: Расписать каждый подход с указанием литературы

\section{Архитектуры нейросетей для различных форматов CAPTCHA}

В данной работе рассмотрены три наиболее популярные и частовстречающиеся 
реализации CAPTCHA, которые применяются для защиты web-ресурсов: аудио CAPTCHA, 
текстовые CAPTCHA и графические CAPTCHA (CAPTCHA с изображениями). Для каждой 
реализации необходим свой подход к решению, разный набор инструментов и библиотек.

Далее для каждой из реализаций будцт рассмотрены различные архитектуры нейронных 
сетей, которые могут быть использованы для автоматизации решения CAPTCHA.

\textbf{Архитектуры нейронных сетей для аудио CAPTCHA}

Для аудио CAPTCHA доступны следующие архитектуры нейронных сетей и инструменты:

\begin{enumerate}
    \item Открытые API для работы с языковыми моделями от Google, Microsoft и 
    других;
    \item Написать еще несколько вариантов...
\end{enumerate}

% TODO: Более подпробно рассмотреть каждый из вариантов с привлечением литературы и так далее

\textbf{Архитектуры нейронных сетей для текстовых CAPTCHA}

Для задачи решения текстовых CAPTCHA могут быть использованы различные модели 
нейронных сетей, которые поддерживают обработку последовательностей различной 
длины. Среди таких архитектур и инструментов можно выделить следующие:

\begin{enumerate}
    \item Tesseract OCR;
    \item CRNN + CTC;
    \item Sequence-to-Sequence.
\end{enumerate}

% TODO: Добавить больше теоретической информации с привлечением литературы

\textbf{Архитектуры нейронных сетей для графических CAPTCHA}

При решении графических CAPTCHA важными являются возможности модели по детекции
и сегментации объектов, поскольку данные CAPTCHA могут требовать как обычного 
поиска объекта, так и выбора клеток, в которых содержится объект. Для решения 
данныхзадач могут применяться следующие инструменты и архитектуры нейронных сетей:

\begin{enumerate}
    \item YOLO;
    \item DETR;
    \item Faster R-CNN.
\end{enumerate}

% TODO: Добавить больше теоретической информации с привлечением литературы

\section{Подготовка и аннотация датасетов}

\textbf{Подготовка датасета для тектcовых CAPTCHA}

Поскольку в открытом доступе отсутствует достаточное количество данных для 
формирования сбалансированного датасета, было принято решение о генерации 
синтетических изображений с использованием специализированных библиотек. В 
качестве основного инструмента выбрана библиотека captcha на языке Python, 
обладающая необходимым функционалом для создания изображений CAPTCHA с 
заданными параметрами. Данная библиотека поддерживает генерацию изображений с 
пользовательскими шрифтами и различными эффектами искажений, что исключает 
необходимость привлечения дополнительных инструментов.

Исходный код генератора синтетических CAPTCHA представлен в приложении~
\ref{code:gen-dataset}.

После создания изображений все они прошли этапы предобработки, направленные на 
улучшение качества данных и повышение эффективности обучения модели. 
Предобработка включала следующие этапы:

\begin{enumerate}
    \item преобразование изображений в градации серого для уменьшения количества 
    каналов и снижения вычислительной нагрузки;
    \item бинаризация изображений с целью получения контрастного представления 
    символов (белый текст на черном фоне);
    \item удаление шумов и фона с использованием морфологических операций, в 
    частности, дилатации.
\end{enumerate}

Исходный код обработчика изображений представлен в приложении~
\ref{code:preprocessing}.

Примеры сгенерированных и предобработанных CAPTCHA приведены на рисунке ниже:

\begin{figure}[H]
    \centering
    \begin{minipage}[h]{0.45\linewidth}
        \center{\includegraphics[width=1\linewidth]{imgs/textcaptcha/YKQ9.png}} 
        \\ а)
    \end{minipage}
    \begin{minipage}[h]{0.45\linewidth}
        \center{\includegraphics[width=1\linewidth]{imgs/textcaptcha/out.png}} 
        \\ б)
    \end{minipage}
    \caption{Изображения CAPTCHA: а) -- сгенерированное изображение, б) -- 
    результат обработки.}
    \label{fig:example-captcha}
\end{figure}

\vspace{-0.7cm}

\textbf{Подготовка датасета для CAPTCHA с изображениями}

Большинство предобученных моделей компьютерного зрения, таких как YOLOv8, обучены 
на датасете COCO~\cite{COCO}, содержащем изображения высокого качества с чёткими 
контурами и однозначной аннотацией объектов. Однако CAPTCHA с изображениями имеют 
принципиально иные характеристики: они могут включать в себя размытие, наложенные 
артефакты, искажения, шумы, повторяющиеся элементы и искусственно пониженное 
разрешение. Всё это снижает эффективность использования стандартных датасетов и 
моделей, не адаптированных под такие условия.

Для обеспечения высокой точности в задаче автоматического решения CAPTCHA 
необходимо подготовить собственный набор данных, приближённый к реальным условиям 
использования. Наиболее эффективным методом является автоматизированный парсинг 
изображений CAPTCHA, представленных на веб-сайтах, использующих визуальные 
CAPTCHA-решения, такие как Google reCAPTCHA v2.

Использование реальных CAPTCHA, собранных в автоматическом режиме, имеет ряд 
преимуществ по сравнению с синтетической генерацией данных:

\begin{enumerate}
    \item изображения содержат разнообразные сцены, освещение, углы обзора и 
    уровни шума, что положительно влияет на способность модели к обобщению;
    \item присутствует большое количество уникальных объектов на фоне, в том 
    числе в частично перекрытых и смазанных вариантах;
    \item отсутствует необходимость в ручной генерации изображений и создании 
    дополнительных искажений для повышения реалистичности;
    \item возможно извлекать текстовые инструкции к CAPTCHA, что позволяет 
    соотносить каждое изображение с требуемым классом.
\end{enumerate}

Для парсинга CAPTCHA был реализован автоматизированный сценарий взаимодействия с 
браузером с использованием библиотеки Selenium~\cite{Selenium}. Данный подход 
позволяет воспроизвести действия пользователя при работе с CAPTCHA, обходя при 
этом ручной ввод. Для обеспечения стабильной работы и масштабируемости процесса 
применялась браузерная автоматизация через WebDriver (в частности, ChromeDriver).

Функциональность парсера включает следующие ключевые этапы:

\begin{enumerate}
    \item поиск iframe-элемента, содержащего чекбокс <<Я не робот>>, и эмуляция 
    клика по нему для инициирования визуальной CAPTCHA;
    \item ожидание загрузки CAPTCHA и извлечение изображения с заданием (включая 
    его URL или пиксельный снимок);
    \item извлечение информации о структуре сетки (количество строк и столбцов), 
    на которую разбито изображение CAPTCHA;
    \item получение текста задания, содержащего имя объекта (например, <<выберите 
    все изображения с мотоциклами>>), для последующего использования в аннотации 
    данных.
\end{enumerate}

Типичная CAPTCHA представляет собой изображение, разделённое на сетку из 3×3 или 
4×4 ячеек, каждая из которых может содержать фрагмент сцены. При этом 
пользователю предлагается выбрать ячейки, в которых присутствует объект заданного 
класса. Процесс парсинга может быть представлена блок-схемой на рис.~
\ref{fig:captcha-flow}.

\begin{figure}[H]
    \centering
    \includegraphics[width=0.6\textwidth]{
        imgs/imagecaptcha/image_captcha_flow.png
    }
    \caption{Блок-схема процесса парсинга CAPTCHA.}
    \label{fig:captcha-flow}
\end{figure}
\vspace{-0.5cm}

Полученные изображения и метаданные (включая текст задания и параметры сетки) 
используются для формирования обучающего датасета, пригодного для дообучения 
модели YOLOv8 в задачах классификации и сегментации объектов.

После получения достаточного количества изображений для составления датасета 
необходимо провести их предварительную обработку и разметку. Это один из 
самыхважных этапов работы, поскольку от качества разметки напрямую зависит 
точность и эффективность последующей работы модели.

Для корректной работы модели YOLO требуется создать иерархическую структуру 
папок, в которой изображения и соответствующие метки будут разделены на 
тренировочную и валидационную выборки. Стандартная структура включает следующие 
директории:

\begin{enumerate}
    \item Директория train -- содержит тренировочную выборку:
    \begin{enumerate}
        \item images -- изображения;
        \item labels -- метки к изображениям.
    \end{enumerate}
    \item Директория val -- содержит валидационную выборку:
    \begin{enumerate}
        \item images -- изображения;
        \item labels -- метки к изображениям.
    \end{enumerate}
\end{enumerate}

Набор классов, пути к выборкам и параметры конфигурации задаются в YAML-файле, 
который передается при обучении модели. Содержимое такого файла для данной 
модели:

\begin{code}
    \captionof{listing}{
        \label{code:train-captcha}Параметры конфигурации для обучения модели
    }
    \vspace{-0.75cm}
    {\small
        \inputminted[mathescape,linenos,frame=lines,breaklines]{yaml}{code/imagecaptcha/train_captcha.yaml}
    }
\end{code}
\vspace{-0.4cm}

Для создания меток используется инструмент CVAT (Computer Vision Annotation Tool) 
-- многофункциональное веб-приложение с поддержкой аннотации объектов с помощью 
полигонов, прямоугольников и других форм. CVAT позволяет экспортировать разметку 
напрямую в формат, совместимый с YOLO~\cite{CVAT}.

Поскольку CAPTCHA-изображения часто содержат объекты с нечёткими контурами, 
наложением и визуальными искажениями, особенно важно использовать ручную точную 
разметку, а не ограничиваться автоматическими методами. Выделение объектов должно 
проводиться как можно точнее, с учётом геометрии контуров. На рисунке ниже 
представлен пример изображения с размеченными объектами:

\begin{figure}[H]
    \centering
    \includegraphics[width=0.9\linewidth]{imgs/imagecaptcha/captcha-poligons.png}
    \caption{Пример разметки изображения с тестовой CAPTCHA.}
    \label{fig:mask-captcha}
\end{figure}
\vspace{-0.5cm}

Кроме того, разметка позволяет учесть сразу несколько объектов разных классов на 
одном изображении, что особенно характерно для CAPTCHA, где в одной сетке могут 
одновременно находиться, например, автомобили и автобусы. Такой подход 
положительно влияет на обобщающую способность модели.

В случае, если количество данных по отдельным классам окажется недостаточным, 
можно дополнительно использовать методы аугментации: вращение, масштабирование, 
искажение цвета и контраста. Однако при хорошо организованном парсинге и разметке 
зачастую удается обойтись без аугментации.


% Глава (возможно с какими-то исследованиями)

\chapter*{Заключение}
\addcontentsline{toc}{chapter}{Заключение}

В ходе выполнения выпускной квалификационной работы были решены следующие задачи:

\begin{enumerate}
    \item Подобраны компоненты, необходимые для прототипирования генератора персонажей.
    \item Спроектированы функциональные возможности приложения.
    \item Реализованы в программном коде алгоритмы согласно базовым правилам D\&D5.
    \item Отлажены и протестированы работоспособность программной и аппаратной части прототипа устройства.
    \item Собран прототипа устройства.
\end{enumerate}

В результате выполнения работы было спроектировано и разработано устройство для генерации персонажа D\&D и все поставленные задачи выпускной квалификационной были решены в полном объеме.

\newpage
\addcontentsline{toc}{chapter}{Список использованной литературы}
\printbibliography[title={Список использованной литературы}]

\newpage
\chapter*{\begin{flushright}Приложение\end{flushright}}
\addcontentsline{toc}{chapter}{Приложение}

\vspace{-0.85cm}\begin{code}
\captionof*{listing}{\label{code:main}Исходный код приложения}
\vspace{0.5cm}
{\small
\inputminted[mathescape,linenos,frame=lines,breaklines]{C++}{main.c}
}
\end{code}

\makelastpage

\end{document}
