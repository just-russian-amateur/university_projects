\documentclass[12pt,a4paper]{article}
\linespread{1,15}
\usepackage[utf8]{inputenc}
\usepackage[T2A]{fontenc}
\usepackage[russian]{babel}
\usepackage{amsmath}
\usepackage{amsfonts}
\usepackage{amssymb}
\usepackage{graphicx}
\usepackage{listings}
\usepackage[title,titletoc]{appendix}
\usepackage[left=30mm, top=20mm, right=15mm, bottom=20mm, nohead, footskip=10mm]{geometry}
\usepackage{indentfirst}

\begin{document}

% \maketitle

\section{Актуальность}

Здравствуйте уважаемые члены комиссии.

Быстро развивающаяся область развлечений -- это область, в которой сочетаются как различные виды искусства, так и программно-аппаратные приложения. В данной работе рассматривается создание программно-аппаратного устройства для Dungeons and Dragons. DnD -- это набор правил для создания игрового персонажа с определенным набором характеристик, которые затем записываются в лист персонажа.

Большинство характеристик рассчитываются путем многократных бросков игральных костей, согласно правилам игры. Данный набор правил является довольно обширным и создание персонажа от начала и до конца превращается в длительный и трудоемкий процесс, который требует от игрока глубокого погружения в правила игры.

% (2 абзаца, примерно 45 сек)

Существуют браузерные приложения для упрощения (автоматизации) создания персонажа. Но у таких реализаций есть на мой взгляд существенный недостаток, наличие или требование постоянного подключения к сети, что является ограничивающим фактором в определенных ситуациях:

\begin{enumerate}
    \item Предпочтения Мастера и игроков;
    \item Игровой процесс происходит в местах, где покрытие слабое.
\end{enumerate}

Исходя из вышесказанного целью моей выпускной квалификационной работы является: 

\section{Цель и задачи}

% (40 сек -- 1 минута)

Спроектировать прототип портативного устройства для генерации персонажа Dungeons \& Dragons. Данное устройство не требует наличие сети Интернет.

Для достижения поставленной цели требуется выполнить следующие задачи:

\begin{enumerate}
    \item Подбор компонентов, необходимых для прототипирования;
    \item Проектирование функциональных возможностей приложения;
    \item Реализация в программном коде алгоритмов согласно базовым правилам D\&D5;
    \item Отладка и тестирование работоспособности программной и аппаратной части прототипа устройства;
    \item Сборка прототипа устройства.
\end{enumerate}

Исходя из поставленных задач были сформированы следующие требования по функционалу программной части:

\section{Необходимый функционал}

\begin{enumerate}
    \item Выбор расы персонажа;
    \item Выбор класса персонажа;
    \item Генерация значений 5 базовых характеристик:
    \begin{enumerate}
        \item Сила (Strong --- сокращенно <<Str>>);   
        \item Телосложение (Constitution --- сокращенно <<Con>>);
        \item Ловкость (Dexterity --- сокращенно <<Dex>>);
        \item Интеллект (Intelligence --- сокращенно <<Int>>);
        \item Мудрость (Wisdom --- сокращенно <<Wis>>);
        \item Харизма (Charisma --- сокращенно <<Cha>>).
    \end{enumerate}
    \item Расчет значений для ряда побочных характеристик: класс защиты и хит-поинтов персонажа (на основе базовых характеристик);
    % \item Выбор снаряжения на основе ранее выбранных игроком характеристик;
    % \item Еще некоторые побочные характеристики.
\end{enumerate}

Указанный функционал является минимально необходимым для генерации персонажа с базовыми характеристиками и может быть расширен для более глубокой проработки персонажа.

Исходя из данного программного функционала были подобраны следующие компоненты для реализации аппаратной части.

\section{Выбор аппаратной части}

Чтобы реализовать прототип были рассмотрены отладочные платы:

\begin{enumerate}
    \item Arduino Uno;
    \item ESP32 38Pin.
\end{enumerate}

На данном слайде представлено сравнение микроконтроллеров, которые находятся в данных отладочных платах.

Отладочная плата Arduino Uno была выбрана по следующим причинам:

\begin{enumerate}
    \item Не имеет в своем составе модулей Bluetooth и WiFi, наличие которых не требуется для прототипа устройства.
    \item Имеет поддержку множества плат расширения без необходимости дополнительного конструирования.
    \item Частота контроллера и памяти являются достаточными для создания прототипа устройства.
\end{enumerate}

Среди вариантов для выбора периферийных устройств были рассмотрены следующие:

\begin{enumerate}
    \item LCD Keypad Shield (плата расширения для Arduino Uno).
    \item LCD-дисплей и кнопки по отдельности.
\end{enumerate}

Как видно из приведенного сравнения по степени удобства подключения плата расширения сильно выигрывает перед отдельными компонентами, поэтому было принято решение использовать ее для создания прототипа.

Для питания платы, в частности, и устройства, в целом, используется внешний аккумулятор (пауэрбанк).

\section{Блок-схемы}

При включении питания устройство выводится информационное сообщение о назначении устройства (левое верхнее фото). Также на слайде продемонстрированы основные пункты меню. В частности, на правом верхнем фото --- пункт меню для выбора расы персонажа, на левом нижнем фото --- пункт меню для выбора класса персонажа (оба этих пункта меню представляют из себя пролистываемый список, между элементами которого пользователь может переключаться с помощью кнопок), на правом нижнем фото --- итоговое меню для просмотра характеристик сгенерированного персонажа.

Рассмотрим блок-схемы генерации персонажа, в частности, обработки выбора расы.

рассказать про схему

На следующем слайде представлен код для приведенной блок-схемы.

После выбора расы производится выбор класса. В пояснительной записке представлена блок-схема для данной функции, на данном слайде вы можете видеть часть кода, которая выполняет данную функцию.

% (по 30 сек на слайд)

Здесь представлена часть блок-схемы и фрагмент кода, в котором описана генерация характеристик персонажа.

\section{Ценовой слайд}

Общие затраты на стоимость комплектующих составили:

\begin{enumerate}
    \item Arduino Uno (отладочная плата) --- 1200 рублей.
    \item LCD Keypad Shield --- 860 рублей.
\end{enumerate}

Для того, чтобы снизить стоимость компонентов конечного устройства предполагается закупать компоненты по отдельности, а не в составе готовых плат, как в случае с прототипом.

На данном слайде представлена стоимость комплектующих. От прототипа к промышленному образцу. Поэтому стоимость комплектующих ниже, без учета зарплаты и прочего.

\section{Заключение}

Задачи достигнуты (если не осталось времени)

Если время осталось, то:

В ходе выполнения работы были решены следующие задачи:

\begin{enumerate}
    \item Подобраны компоненты, необходимые для прототипирования генератора персонажей;
    \item Спроектированы функциональные возможности приложения;
    \item Реализованы в программном коде алгоритмы согласно базовым правилам D\&D5;
    \item Отлажены и протестированы работоспособность программной и аппаратной части прототипа устройства;
    \item Собран прототипа устройства.
\end{enumerate}

В результате выполнения работы было спроектировано и разработано устройство для генерации персонажа D\&D и все поставленные задачи были решены в полном объеме.

\end{document}
