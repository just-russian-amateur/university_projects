\documentclass{altsu-report}
\linespread{1,15}
\title{Автоматизация решения CAPTCHA в формате изображений}
\author{А.\,В.~Лаптев}
\groupnumber{5.306М}
\GradebookNumber{1337}
\supervisor{А.\,В.~Калачев}
\supervisordegree{доц. каф. ВТиЭ}
\ministry{Министерство науки и высшего образования}
\country{Российской Федерации}
\fulluniversityname{ФГБОУ ВО Алтайский государственный университет}
\institute{Институт цифровых технологий, электроники и физики}
\department{Кафедра вычислительной техники и электроники}
\departmentchief{В.\,В.~Пашнев}
\departmentchiefdegree{к.ф.-м.н., доцент}
\shortdepartment{ВТиЭ}
\abstractRU{
Данная работа посвящена разработке и обучению нейросетевой модели для автоматического распознавания объектов на изображениях CAPTCHA. Целью исследования является создание системы, способной интерпретировать задания CAPTCHA и выбирать нужные объекты на изображениях в автоматическом режиме. В рамках проекта был собран собственный датасет CAPTCHA-изображений, проведена их разметка, обучена модель YOLOv8 с поддержкой сегментации и реализован скрипт автоматического взаимодействия с web-интерфейсами. Результаты тестирования модели подтвердили её способность эффективно решать задания CAPTCHA, что делает разработку применимой в задачах автоматизации тестирования и обхода визуальных защит.
}
\keysRU{CAPTCHA, нейронная сеть, YOLOv8, сегментация, компьютерное зрение, Selenium, CVAT}

\date{\the\year}

% Подключение файлов с библиотекой.
\addbibresource{graduate-students.bib}

\begin{document}

\maketitle

\setcounter{page}{2}
\makeabstract
\tableofcontents

\chapter*{Введение}
\addcontentsline{toc}{chapter}{Введение}

В последние годы CAPTCHA (Completely Automated Public Turing test to tell Computers and Humans Apart) остаётся одним из наиболее распространённых способов защиты web-ресурсов от автоматических скриптов. Одной из наиболее сложных форм CAPTCHA являются изображения, содержащие множество объектов с размытыми контурами, шумами и низким разрешением, что затрудняет автоматическое распознавание.

Сложность таких изображений делает их трудными не только для ботов, но и для современных систем компьютерного зрения. При этом в процессе автоматизированного тестирования web-приложений возникает необходимость обхода подобных CAPTCHA, что требует разработки устойчивых и точных методов распознавания визуального контента.

Целью данной работы является разработка и обучение нейронной сети с поддержкой сегментации, способной автоматически распознавать объекты на изображениях CAPTCHA и выполнять задания, формируемые системой защиты.  

Для достижения поставленной цели необходимо решить следующие задачи:
\begin{enumerate}
    \item проанализировать типы CAPTCHA, применяемых на web-ресурсах;
    \item выбрать подходящую архитектуру нейронной сети, обеспечивающую высокую скорость и точность;
    \item собрать и разметить датасет реальных CAPTCHA с изображениями объектов;
    \item провести предварительную обработку изображений и формирование структуры датасета;
    \item обучить выбранную модель на собранных данных;
    \item разработать скрипт для автоматизированного прохождения CAPTCHA с использованием обученной модели;
    \item протестировать модель в реальных условиях и оценить её эффективность.
\end{enumerate}

% Теоретическая глава про все виды CAPTCHA

\chapter{Современные методы защиты от ботов и спама на основе CAPTCHA}

\section{История CAPTCHA}

Проверочный код CAPTCHA -- это метод защиты, основанный на принципе 
аутентификации «вызов-ответ». Он предназначен для предотвращения автоматических 
действий, таких как спам или попытки взлома учетных записей, путем выполнения 
пользователем простого теста, подтверждающего, что он человек, а не программа 
[1].

CAPTCHA является важной мерой безопасности, так как предотвращает автоматические 
атаки, например, массовую регистрацию ботов, и защищает данные пользователя. 
Современные системы CAPTCHA используют не только текст, но и изображения, аудио, 
поведенческие анализы и другие инновационные подходы, чтобы сделать тесты удобными 
для людей, но сложными для программ.

На сегодняшний день наиболее распространенные виды CAPTCHA включают:

\begin{enumerate}
    \item reCAPTCHA -- разработанная Google система, которая предлагает тесты 
    на основе распознавания объектов, анализа поведения или текстовых символов.
    \item hCAPTCHA -- альтернатива reCAPTCHA, фокусирующаяся на защите 
    конфиденциальности пользователей.
    \item Capy -- система CAPTCHA, предлагающая пользователю головоломки, 
    например, сборку изображения или взаимодействие с элементами интерфейса [2].
\end{enumerate}

\section{reCAPTCHA}

reCAPTCHA -- система защиты от автоматизированных действий, разработанная Google, 
которая помогает различать человека и бота. Она объединяет несколько подходов, 
делая проверку удобной для пользователей, но сложной для автоматических систем 
[3].

reCAPTCHA включает в себя следующие версии:

\begin{enumerate}
    \item reCAPTCHA v1 (устарела в 2018 году):
    \begin{enumerate}
        \item пользователи вводили текст, состоящий из искаженных слов, 
        отображаемых на изображении;
        \item использовала слова из книг и документов, которые не могли быть 
        распознаны OCR.
    \end{enumerate}
    \item reCAPTCHA v2:
    \begin{enumerate}
        \item клик по флажку: пользователи подтверждают, что они не роботы, 
        нажимая на флажок «Я не робот»;
        \item выбор объектов на изображениях: пользователи идентифицируют 
        заданные объекты на сетке из картинок;
        \item аудио CAPTCHA: для пользователей с ограничениями зрения, 
        предлагается прослушать запись и ввести услышанные символы.
    \end{enumerate}
    \item reCAPTCHA v3:
    \begin{enumerate}
        \item полностью работает в фоновом режиме, анализируя поведение 
        пользователя на странице;
        \item не требует явных действий, если пользователь считается 
        низкорискованным [4].
    \end{enumerate}
\end{enumerate}

\section{hCAPTCHA}

hCAPTCHA -- это альтернативная система CAPTCHA, разработанная для защиты сайтов 
от ботов и спама, при этом уделяющая особое внимание конфиденциальности 
пользователей. Она стала популярной благодаря своей гибкости и ориентации на 
защиту данных [5].  

Основные особенности hCAPTCHA:

\begin{enumerate}
    \item конфиденциальность:
    \begin{enumerate}
        \item в отличие от reCAPTCHA, hCAPTCHA не собирает данные о 
        пользователях для рекламных целей, что делает ее привлекательной с точки 
        зрения соблюдения конфиденциальности.
    \end{enumerate}
    \item простота интеграции:
    \begin{enumerate}
        \item легко интегрируется с web-сайтами через API;
        \item совместима с большинством популярных платформ, таких как WordPress, 
        и может быть настроена для разных типов взаимодействия.
    \end{enumerate}
    \item модели монетизации:
    \begin{enumerate}
        \item владельцы сайтов могут зарабатывать, разрешая hCAPTCHA 
        использовать проверочные задачи, связанные с машинным обучением, 
        например, разметку данных.
    \end{enumerate}
\end{enumerate}

Виды взаимодействия с пользователями:

\begin{enumerate}
    \item графическая CAPTCHA: выбор изображений, соответствующих запросу;
    \item текстовая CAPTCHA: ввод символов (редко используется);
    \item аудио CAPTCHA: для пользователей с ограниченными возможностями, 
    предлагается прослушать и ввести услышанные символы;
    \item клик CAPTCHA: нажатие на флажок «Я не робот» (для низкорискованных 
    пользователей).
\end{enumerate}

\section{Capy}

Capy CAPTCHA -- это инновационная система CAPTCHA, разработанная с акцентом на 
удобство для пользователей и адаптацию к современным web-средам. Она предлагает 
интерактивные методы проверки, направленные на минимизацию раздражения 
пользователей при сохранении высокого уровня защиты от ботов [6].

Основные особенности Capy CAPTCHA:

\begin{enumerate}
    \item интерактивность:
    \begin{enumerate}
        \item Capy использует методы проверки, которые требуют не просто ввода 
        текста или выбора картинок, а выполнения задач, таких как перемещение 
        объектов;
        \item простые задачи делают процесс проверки менее раздражающим и более 
        интуитивным;
    \end{enumerate}
    \item гибкость настройки:
    \begin{enumerate}
        \item система может быть адаптирована под конкретные нужды сайта, 
        включая выбор сложности задач и дизайн интерфейса.
    \end{enumerate}
    \item доступность:
    \begin{enumerate}
        \item подходит для пользователей с различными потребностями, включая 
        мобильные устройства.
    \end{enumerate}
\end{enumerate}

Виды взаимодействия с пользователями:

\begin{enumerate}
    \item головоломки (Puzzle CAPTCHA): сборка пазла с перемещением недостающих 
    элементов в нужное место;
    \item тесты на логику и распознавание: выбор нужного объекта или 
    логического варианта из предложенных;
    \item текстовая CAPTCHA (редко используется).
\end{enumerate}

Capy CAPTCHA используется на сайтах, где важны как защита от ботов, так и 
положительный пользовательский опыт. Особенно популярна в проектах с высоким 
акцентом на дизайн и пользовательское взаимодействие.

% Глава про разработку всех решателей

\chapter{Методология решения задач CAPTCHA}

\section{Общие подходы к автоматизированному решению CAPTCHA}

Для автоматизации решения CAPTCHA могут применяться различные методы и подходы, 
которые зависят от конкретной реализации CAPTCHA. В рамках данной работы можно 
выделить следующий список таких методов:

\begin{enumerate}
    \item cинтез речи;
    \item сегментация;
    \item классификация;
    \item распознавание последовательности;
    \item может быть продлю...
\end{enumerate}

% TODO: Расписать каждый подход с указанием литературы

\section{Архитектуры нейросетей для различных форматов CAPTCHA}

В данной работе рассмотрены три наиболее популярные и частовстречающиеся 
реализации CAPTCHA, которые применяются для защиты web-ресурсов: аудио CAPTCHA, 
текстовые CAPTCHA и графические CAPTCHA (CAPTCHA с изображениями). Для каждой 
реализации необходим свой подход к решению, разный набор инструментов и библиотек.

Далее для каждой из реализаций будцт рассмотрены различные архитектуры нейронных 
сетей, которые могут быть использованы для автоматизации решения CAPTCHA.

\textbf{Архитектуры нейронных сетей для аудио CAPTCHA}

Для аудио CAPTCHA доступны следующие архитектуры нейронных сетей и инструменты:

\begin{enumerate}
    \item Открытые API для работы с языковыми моделями от Google, Microsoft и 
    других;
    \item Написать еще несколько вариантов...
\end{enumerate}

% TODO: Более подпробно рассмотреть каждый из вариантов с привлечением литературы и так далее

\textbf{Архитектуры нейронных сетей для текстовых CAPTCHA}

Для задачи решения текстовых CAPTCHA могут быть использованы различные модели 
нейронных сетей, которые поддерживают обработку последовательностей различной 
длины. Среди таких архитектур и инструментов можно выделить следующие:

\begin{enumerate}
    \item Tesseract OCR;
    \item CRNN + CTC;
    \item Sequence-to-Sequence.
\end{enumerate}

% TODO: Добавить больше теоретической информации с привлечением литературы

\textbf{Архитектуры нейронных сетей для графических CAPTCHA}

При решении графических CAPTCHA важными являются возможности модели по детекции
и сегментации объектов, поскольку данные CAPTCHA могут требовать как обычного 
поиска объекта, так и выбора клеток, в которых содержится объект. Для решения 
данныхзадач могут применяться следующие инструменты и архитектуры нейронных сетей:

\begin{enumerate}
    \item YOLO;
    \item DETR;
    \item Faster R-CNN.
\end{enumerate}

% TODO: Добавить больше теоретической информации с привлечением литературы

\section{Подготовка и аннотация датасетов}

\textbf{Подготовка датасета для тектcовых CAPTCHA}

Поскольку в открытом доступе отсутствует достаточное количество данных для 
формирования сбалансированного датасета, было принято решение о генерации 
синтетических изображений с использованием специализированных библиотек. В 
качестве основного инструмента выбрана библиотека captcha на языке Python, 
обладающая необходимым функционалом для создания изображений CAPTCHA с 
заданными параметрами. Данная библиотека поддерживает генерацию изображений с 
пользовательскими шрифтами и различными эффектами искажений, что исключает 
необходимость привлечения дополнительных инструментов.

Исходный код генератора синтетических CAPTCHA представлен в приложении~
\ref{code:gen-dataset}.

После создания изображений все они прошли этапы предобработки, направленные на 
улучшение качества данных и повышение эффективности обучения модели. 
Предобработка включала следующие этапы:

\begin{enumerate}
    \item преобразование изображений в градации серого для уменьшения количества 
    каналов и снижения вычислительной нагрузки;
    \item бинаризация изображений с целью получения контрастного представления 
    символов (белый текст на черном фоне);
    \item удаление шумов и фона с использованием морфологических операций, в 
    частности, дилатации.
\end{enumerate}

Исходный код обработчика изображений представлен в приложении~
\ref{code:preprocessing}.

Примеры сгенерированных и предобработанных CAPTCHA приведены на рисунке ниже:

\begin{figure}[H]
    \centering
    \begin{minipage}[h]{0.45\linewidth}
        \center{\includegraphics[width=1\linewidth]{imgs/textcaptcha/YKQ9.png}} 
        \\ а)
    \end{minipage}
    \begin{minipage}[h]{0.45\linewidth}
        \center{\includegraphics[width=1\linewidth]{imgs/textcaptcha/out.png}} 
        \\ б)
    \end{minipage}
    \caption{Изображения CAPTCHA: а) -- сгенерированное изображение, б) -- 
    результат обработки.}
    \label{fig:example-captcha}
\end{figure}

\vspace{-0.7cm}

\textbf{Подготовка датасета для CAPTCHA с изображениями}

Большинство предобученных моделей компьютерного зрения, таких как YOLOv8, обучены 
на датасете COCO~\cite{COCO}, содержащем изображения высокого качества с чёткими 
контурами и однозначной аннотацией объектов. Однако CAPTCHA с изображениями имеют 
принципиально иные характеристики: они могут включать в себя размытие, наложенные 
артефакты, искажения, шумы, повторяющиеся элементы и искусственно пониженное 
разрешение. Всё это снижает эффективность использования стандартных датасетов и 
моделей, не адаптированных под такие условия.

Для обеспечения высокой точности в задаче автоматического решения CAPTCHA 
необходимо подготовить собственный набор данных, приближённый к реальным условиям 
использования. Наиболее эффективным методом является автоматизированный парсинг 
изображений CAPTCHA, представленных на веб-сайтах, использующих визуальные 
CAPTCHA-решения, такие как Google reCAPTCHA v2.

Использование реальных CAPTCHA, собранных в автоматическом режиме, имеет ряд 
преимуществ по сравнению с синтетической генерацией данных:

\begin{enumerate}
    \item изображения содержат разнообразные сцены, освещение, углы обзора и 
    уровни шума, что положительно влияет на способность модели к обобщению;
    \item присутствует большое количество уникальных объектов на фоне, в том 
    числе в частично перекрытых и смазанных вариантах;
    \item отсутствует необходимость в ручной генерации изображений и создании 
    дополнительных искажений для повышения реалистичности;
    \item возможно извлекать текстовые инструкции к CAPTCHA, что позволяет 
    соотносить каждое изображение с требуемым классом.
\end{enumerate}

Для парсинга CAPTCHA был реализован автоматизированный сценарий взаимодействия с 
браузером с использованием библиотеки Selenium~\cite{Selenium}. Данный подход 
позволяет воспроизвести действия пользователя при работе с CAPTCHA, обходя при 
этом ручной ввод. Для обеспечения стабильной работы и масштабируемости процесса 
применялась браузерная автоматизация через WebDriver (в частности, ChromeDriver).

Функциональность парсера включает следующие ключевые этапы:

\begin{enumerate}
    \item поиск iframe-элемента, содержащего чекбокс <<Я не робот>>, и эмуляция 
    клика по нему для инициирования визуальной CAPTCHA;
    \item ожидание загрузки CAPTCHA и извлечение изображения с заданием (включая 
    его URL или пиксельный снимок);
    \item извлечение информации о структуре сетки (количество строк и столбцов), 
    на которую разбито изображение CAPTCHA;
    \item получение текста задания, содержащего имя объекта (например, <<выберите 
    все изображения с мотоциклами>>), для последующего использования в аннотации 
    данных.
\end{enumerate}

Типичная CAPTCHA представляет собой изображение, разделённое на сетку из 3×3 или 
4×4 ячеек, каждая из которых может содержать фрагмент сцены. При этом 
пользователю предлагается выбрать ячейки, в которых присутствует объект заданного 
класса. Процесс парсинга может быть представлена блок-схемой на рис.~
\ref{fig:captcha-flow}.

\begin{figure}[H]
    \centering
    \includegraphics[width=0.6\textwidth]{
        imgs/imagecaptcha/image_captcha_flow.png
    }
    \caption{Блок-схема процесса парсинга CAPTCHA.}
    \label{fig:captcha-flow}
\end{figure}
\vspace{-0.5cm}

Полученные изображения и метаданные (включая текст задания и параметры сетки) 
используются для формирования обучающего датасета, пригодного для дообучения 
модели YOLOv8 в задачах классификации и сегментации объектов.

После получения достаточного количества изображений для составления датасета 
необходимо провести их предварительную обработку и разметку. Это один из 
самыхважных этапов работы, поскольку от качества разметки напрямую зависит 
точность и эффективность последующей работы модели.

Для корректной работы модели YOLO требуется создать иерархическую структуру 
папок, в которой изображения и соответствующие метки будут разделены на 
тренировочную и валидационную выборки. Стандартная структура включает следующие 
директории:

\begin{enumerate}
    \item Директория train -- содержит тренировочную выборку:
    \begin{enumerate}
        \item images -- изображения;
        \item labels -- метки к изображениям.
    \end{enumerate}
    \item Директория val -- содержит валидационную выборку:
    \begin{enumerate}
        \item images -- изображения;
        \item labels -- метки к изображениям.
    \end{enumerate}
\end{enumerate}

Набор классов, пути к выборкам и параметры конфигурации задаются в YAML-файле, 
который передается при обучении модели. Содержимое такого файла для данной 
модели:

\begin{code}
    \captionof{listing}{
        \label{code:train-captcha}Параметры конфигурации для обучения модели
    }
    \vspace{-0.75cm}
    {\small
        \inputminted[mathescape,linenos,frame=lines,breaklines]{yaml}{code/imagecaptcha/train_captcha.yaml}
    }
\end{code}
\vspace{-0.4cm}

Для создания меток используется инструмент CVAT (Computer Vision Annotation Tool) 
-- многофункциональное веб-приложение с поддержкой аннотации объектов с помощью 
полигонов, прямоугольников и других форм. CVAT позволяет экспортировать разметку 
напрямую в формат, совместимый с YOLO~\cite{CVAT}.

Поскольку CAPTCHA-изображения часто содержат объекты с нечёткими контурами, 
наложением и визуальными искажениями, особенно важно использовать ручную точную 
разметку, а не ограничиваться автоматическими методами. Выделение объектов должно 
проводиться как можно точнее, с учётом геометрии контуров. На рисунке ниже 
представлен пример изображения с размеченными объектами:

\begin{figure}[H]
    \centering
    \includegraphics[width=0.9\linewidth]{imgs/imagecaptcha/captcha-poligons.png}
    \caption{Пример разметки изображения с тестовой CAPTCHA.}
    \label{fig:mask-captcha}
\end{figure}
\vspace{-0.5cm}

Кроме того, разметка позволяет учесть сразу несколько объектов разных классов на 
одном изображении, что особенно характерно для CAPTCHA, где в одной сетке могут 
одновременно находиться, например, автомобили и автобусы. Такой подход 
положительно влияет на обобщающую способность модели.

В случае, если количество данных по отдельным классам окажется недостаточным, 
можно дополнительно использовать методы аугментации: вращение, масштабирование, 
искажение цвета и контраста. Однако при хорошо организованном парсинге и разметке 
зачастую удается обойтись без аугментации.

% Глава (возможно с какими-то исследованиями)
\chapter{Обучение и тестирование модели YOLOv8 на реальных CAPTCHA}

\section{Обучение модели}

В качестве основной архитектуры была выбрана модель YOLOv8m-seg, поддерживающая сегментацию объектов. Она представляет собой сбалансированное решение между качеством распознавания, производительностью и требованиями к аппаратному обеспечению. Благодаря своей универсальности, модель подходит как для задач классификации, так и для задач детектирования и сегментации, что особенно важно при работе с CAPTCHA, содержащими зашумлённые или плохо различимые объекты.

Преимущества YOLOv8m-seg заключаются в следующем:

\begin{enumerate}
    \item наличие встроенной поддержки сегментации объектов, что особенно важно при необходимости выделения фрагментов изображений;
    \item возможность использования предобученных весов, сокращающих время на обучение и повышающих стартовую точность;
    \item высокая скорость инференса по сравнению с другими моделями сегментации (например, Mask R-CNN или DETR);
    \item встроенные средства аугментации (изменения яркости, повороты, масштабирование и пр.);
    \item удобный интерфейс через библиотеку ultralytics, позволяющий быстро запускать обучение, логировать метрики и визуализировать результаты;
    \item полная совместимость с аннотациями в формате YOLO, полученными из CVAT.
\end{enumerate}

Перед запуском обучения структура данных была организована в соответствии с требованиями YOLOv8: директории train и val содержали соответствующие изображения и файлы разметки, а в .yaml файле конфигурации были указаны пути к выборкам и список классов.

Обучение проводилось на 35 эпохах при размере изображений 640×640 пикселей и размере батча 8. Использование предобученных весов позволило достичь стабильного снижения функции потерь с первых эпох, а встроенные механизмы аугментации способствовали улучшению обобщающей способности модели.

Результаты обучения отслеживались по ключевым метрикам (IoU, Precision, Recall, Loss), которые визуализировались автоматически. Примеры графиков с результатами обучения приведены ниже:

\begin{figure}[H]
    \centering
    \includegraphics[width=1\linewidth]{imgs/results.png}
    \caption{Изменение ключевых метрик в процессе обучения.}
    \label{fig:metrics}
\end{figure}
\vspace{-0.5cm}

Также, была построена нормализованная матрица ошибок для определения точности предсказания необходимых классов на валидационной выборке, которая представлена на рис.~\ref{fig:confusion}.

\begin{figure}[H]
    \centering
    \includegraphics[width=1\linewidth]{imgs/confusion_matrix_normalized.png}
    \caption{Матрица ошибок для изображений валидационной выборки.}
    \label{fig:confusion}
\end{figure}
\vspace{-0.5cm}

\section{Тестирование модели}

После завершения обучения модель была протестирована на реальных CAPTCHA, собранных с помощью автоматического парсера, реализованного на базе библиотеки Selenium. Тестирование проводилось в автоматическом режиме, имитируя реальные действия пользователя в браузере, что позволило оценить работоспособность системы в условиях, приближенных к реальной эксплуатации.

Сценарий тестирования предусматривал выполнение следующих шагов:

\begin{enumerate}
    \item автоматический переход к странице с CAPTCHA и активация чекбокса <<Я не робот>>;
    \item извлечение изображения CAPTCHA (включая структуру сетки и текст задания);
    \item определение целевого объекта из текста задания (например, «выберите все изображения с автобусами»);
    \item разбиение изображения CAPTCHA на ячейки (в зависимости от размера сетки — 3×3 или 4×4);
    \item применение обученной модели для сегментации и классификации каждого изображения или фрагмента;
    \item определение ячеек, содержащих нужный класс, и программная симуляция кликов по ним;
    \item повторная попытка прохождения CAPTCHA в случае, если результат оказался некорректным (что также фиксировалось в логах).
\end{enumerate}

Тестирование было организовано в виде цикла, позволяющего автоматически проходить CAPTCHA до тех пор, пока не будет достигнут положительный результат. Это позволило зафиксировать частоту ошибок модели и определить случаи, в которых требуются дообучение или оптимизация.

Рабочий процесс тестирования и взаимодействия модели с CAPTCHA представлен на блок-схеме ниже.

\begin{figure}[H]
    \centering
    \includegraphics[width=1\textwidth]{imgs/solve_captcha_flow.png}
    \caption{Блок-схема процесса прохождения CAPTCHA.}
    \label{fig:solve-captcha}
\end{figure}
\vspace{-0.5cm}

Полученные данные используются для последующего анализа качества модели и корректировки процесса обучения. Основное внимание при анализе будет уделено типам ошибок, сложности распознаваемых объектов и влиянию качества исходного изображения на точность сегментации.


\chapter*{Заключение}
\addcontentsline{toc}{chapter}{Заключение}

В рамках данной работы была реализована система автоматического распознавания и прохождения CAPTCHA с изображениями, основанная на использовании нейросетевой модели YOLOv8 с поддержкой сегментации. 

В ходе исследования были решены следующие задачи:
\begin{enumerate}
    \item проанализированы основные форматы CAPTCHA и выделены ключевые требования к модели;
    \item выбран подходящий вариант модели -- YOLOv8m-seg, обеспечивающий баланс между точностью и производительностью;
    \item собран и размечен собственный датасет CAPTCHA-изображений с использованием инструментов Selenium и CVAT;
    \item реализована и протестирована система обучения и предобработки данных;
    \item создан скрипт автоматизированного взаимодействия с CAPTCHA в браузере;
    \item проведено тестирование модели в реальных условиях, продемонстрировавшее достаточную точность распознавания объектов и успешность прохождения CAPTCHA.
\end{enumerate}

Результаты работы подтверждают применимость современных моделей компьютерного зрения для решения задач, связанных с автоматизацией взаимодействия с защищёнными web-ресурсами. Разработанная система может использоваться как в рамках автоматического тестирования web-интерфейсов, так и в исследованиях в области распознавания сложных визуальных паттернов.

\newpage
\addcontentsline{toc}{chapter}{Список использованной литературы}
\printbibliography[title={Список использованной литературы}]

\newpage
\chapter*{\begin{flushright}Приложение\end{flushright}}
\addcontentsline{toc}{chapter}{Приложение}

\vspace{-0.85cm}
\begin{code}
\captionof{listing}{\label{code:get-captcha}Исходный код получения CAPTCHA с целевого сайта}
\vspace{0.5cm}
{\small
\inputminted[mathescape,linenos,frame=lines,breaklines]{Python}{code/get_captcha.py}
}
\end{code}

\begin{code}
\captionof{listing}{\label{code:recognize}Исходный код дообучения модели на датасете}
\vspace{0.5cm}
{\small
\inputminted[mathescape,linenos,frame=lines,breaklines]{Python}{code/recognize_objects.py}
}
\end{code}

\begin{code}
\captionof{listing}{\label{code:solve-captcha}Исходный код автоматизированного решения CAPTCHA на сайте}
\vspace{0.5cm}
{\small
\inputminted[mathescape,linenos,frame=lines,breaklines]{Python}{code/solve_captcha.py}
}
\end{code}

\end{document}
