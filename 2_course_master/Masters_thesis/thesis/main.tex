\documentclass[14pt, oneside]{altsu-bachelor}

\title{Исследование и реализация методов автоматического распознавания CAPTCHA различных форматов на основе нейросетевых моделей}
\author{А.\,В.~Лаптев}
\groupnumber{5.306M}
\GradebookNumber{7}
\supervisor{А.\,В.~Калачев}
\supervisordegree{к.ф.-м.н., доцент}
\ministry{Министерство науки и высшего образования}
\country{Российской Федерации}
\fulluniversityname{ФГБОУ ВО Алтайский государственный университет}
\institute{Институт цифровых технологий, электроники и физики}
\department{Кафедра вычислительной техники и электроники}
\departmentchief{В.\,В.~Пашнев}
\departmentchiefdegree{к.ф.-м.н., доцент}
\shortdepartment{ВТиЭ}
\ChairmanOfTheStateCertificationCommission{С.\,П.~Пронин}
\ChairmanOfTheStateCertificationCommissiondegree{д.т.н., проф.}
\NormController{В.\,В.~Белозерских}
\NormControllerdegree{ст. пр.}
\Consultant{}
\Consultantdegree{}
\UDC{004.89}
\docname{МР 09.04.01}
\abstractRU{
    В данной работе рассматривается проблема автоматического распознавания 
    CAPTCHA -- технологий, предназначенных для различения действий человека и 
    компьютера. Целью данной работы является разработка и тестирование 
    универсального программного решения для автоматизированного распознавания 
    CAPTCHA в различных форматах: текстовых, графических и аудиоформатах.
    
    В работе проведён обзор существующих подходов к распознаванию различных типов 
    CAPTCHA, с использованием нейросетевых архитектур и специализированные API 
    для обработки мультимедийной информации. В качестве инструментов 
    использовались: YOLOv8 -- для анализа изображений и графических CAPTCHA, 
    Google Web Speech API -- для обработки и расшифровки аудиофайлов, содержащих 
    голосовые CAPTCHA, модель последовательного обучения (Sequence-to-Sequence) 
    -- для текстовых задач CAPTCHA. Автоматизация тестирования решений на 
    реальных web-страницах осуществлялась с использованием инструментов 
    автоматизации работы с браузером.
    
    Результаты экспериментов показали достаточную точность распознавания для всех 
    исследованных форматов CAPTCHA. Было определено, что комбинация современных 
    нейросетевых методов с предварительной обработкой аудио- и графических данных 
    позволяет эффективно обходить большинство популярных CAPTCHA-систем.
}
% \abstractEN{Большой текст на английском!}
\keysRU{
    CAPTCHA, нейронные сети, распознавание речи, распознавание текста, 
    распознавание объектов на изображении, Sequence-to-Sequence, YOLO, Google Web 
    Speech API, автоматизация
}
% \keysEN{computer simulation, distributed version control}
\countWorkPage{86}
\countWorkImg{12}
\countWorkLit{53}
\countWorkTab{1}

\date{\the\year}

% Подключение файлов с библиотекой.
\addbibresource{graduate-students.bib}

\begin{document}
\maketitle

\setcounter{page}{2}
\makeabstract
\tableofcontents

\chapter*{ВВЕДЕНИЕ}
\addcontentsline{toc}{chapter}{ВВЕДЕНИЕ}

С развитием цифровых технологий и ростом интернет-активности существенно возросла 
потребность в защите web-ресурсов от автоматизированного взаимодействия. Одним 
из ключевых инструментов такой защиты являются системы CAPTCHA (Completely 
Automated Public Turing test to tell Computers and Humans Apart), задача которых 
-- отличить действия человека от автоматического скрипта~\cite{captchawiki}. 
CAPTCHA применяется для предотвращения спама, злоупотреблений при регистрации, 
массовых запросов к сервисам и подобных форм мошеннической активности.

Современные системы CAPTCHA предлагают множество форматов: текстовые (с 
искажённым символьным изображением), графические (выбор изображений по 
заданному критерию), а также аудио (воспроизведение и распознавание голосовой 
записи в условиях шумов). Одновременно с этим появляются возможности для их 
автоматического распознавания, в том числе с использованием методов 
машинного обучения и нейросетевых архитектур.

Актуальность данной работы обусловлена как возрастающей сложностью 
CAPTCHA-систем, так и развитием инструментов, позволяющих преодолевать защитные 
механизмы web-ресурсов. Анализ эффективности и разработка подходов для 
автоматизированного решения CAPTCHA могут применяться не только с точки зрения 
изучения устойчивости самих систем, но и в рамках исследования прикладного 
применения нейросетевых моделей в задачах распознавания информации в условиях 
ограничений~\cite{captchatrouble2}.

Целью данной работы является разработка и анализ комплексного решения к 
автоматизации решения CAPTCHA в различных форматах с использованием современных 
нейросетевых инструментов и API для распознавания.

Для достижения поставленной цели были сформулированы следующие задачи:

\begin{enumerate}
    \item провести обзор существующих форматов CAPTCHA и методов их защиты;
    \item разработать систему автоматического распознавания текстовых CAPTCHA с 
    искажениями;
    \item реализовать подход к решению графических CAPTCHA на основе методов 
    компьютерного зрения и нейросетевых моделей;
    \item построить решение для аудио CAPTCHA с использованием средств 
    автоматического распознавания речи;
    \item протестировать реализованные решения в реальных условиях, оценить 
    точность распознавания и стабильность работы.
\end{enumerate}

% Теоретическая глава про все виды CAPTCHA

\chapter{Современные методы защиты от ботов и спама на основе CAPTCHA}

\section{История CAPTCHA}

Проверочный код CAPTCHA -- это метод защиты, основанный на принципе 
аутентификации «вызов-ответ». Он предназначен для предотвращения автоматических 
действий, таких как спам или попытки взлома учетных записей, путем выполнения 
пользователем простого теста, подтверждающего, что он человек, а не программа 
[1].

CAPTCHA является важной мерой безопасности, так как предотвращает автоматические 
атаки, например, массовую регистрацию ботов, и защищает данные пользователя. 
Современные системы CAPTCHA используют не только текст, но и изображения, аудио, 
поведенческие анализы и другие инновационные подходы, чтобы сделать тесты удобными 
для людей, но сложными для программ.

На сегодняшний день наиболее распространенные виды CAPTCHA включают:

\begin{enumerate}
    \item reCAPTCHA -- разработанная Google система, которая предлагает тесты 
    на основе распознавания объектов, анализа поведения или текстовых символов.
    \item hCAPTCHA -- альтернатива reCAPTCHA, фокусирующаяся на защите 
    конфиденциальности пользователей.
    \item Capy -- система CAPTCHA, предлагающая пользователю головоломки, 
    например, сборку изображения или взаимодействие с элементами интерфейса [2].
\end{enumerate}

\section{reCAPTCHA}

reCAPTCHA -- система защиты от автоматизированных действий, разработанная Google, 
которая помогает различать человека и бота. Она объединяет несколько подходов, 
делая проверку удобной для пользователей, но сложной для автоматических систем 
[3].

reCAPTCHA включает в себя следующие версии:

\begin{enumerate}
    \item reCAPTCHA v1 (устарела в 2018 году):
    \begin{enumerate}
        \item пользователи вводили текст, состоящий из искаженных слов, 
        отображаемых на изображении;
        \item использовала слова из книг и документов, которые не могли быть 
        распознаны OCR.
    \end{enumerate}
    \item reCAPTCHA v2:
    \begin{enumerate}
        \item клик по флажку: пользователи подтверждают, что они не роботы, 
        нажимая на флажок «Я не робот»;
        \item выбор объектов на изображениях: пользователи идентифицируют 
        заданные объекты на сетке из картинок;
        \item аудио CAPTCHA: для пользователей с ограничениями зрения, 
        предлагается прослушать запись и ввести услышанные символы.
    \end{enumerate}
    \item reCAPTCHA v3:
    \begin{enumerate}
        \item полностью работает в фоновом режиме, анализируя поведение 
        пользователя на странице;
        \item не требует явных действий, если пользователь считается 
        низкорискованным [4].
    \end{enumerate}
\end{enumerate}

\section{hCAPTCHA}

hCAPTCHA -- это альтернативная система CAPTCHA, разработанная для защиты сайтов 
от ботов и спама, при этом уделяющая особое внимание конфиденциальности 
пользователей. Она стала популярной благодаря своей гибкости и ориентации на 
защиту данных [5].  

Основные особенности hCAPTCHA:

\begin{enumerate}
    \item конфиденциальность:
    \begin{enumerate}
        \item в отличие от reCAPTCHA, hCAPTCHA не собирает данные о 
        пользователях для рекламных целей, что делает ее привлекательной с точки 
        зрения соблюдения конфиденциальности.
    \end{enumerate}
    \item простота интеграции:
    \begin{enumerate}
        \item легко интегрируется с web-сайтами через API;
        \item совместима с большинством популярных платформ, таких как WordPress, 
        и может быть настроена для разных типов взаимодействия.
    \end{enumerate}
    \item модели монетизации:
    \begin{enumerate}
        \item владельцы сайтов могут зарабатывать, разрешая hCAPTCHA 
        использовать проверочные задачи, связанные с машинным обучением, 
        например, разметку данных.
    \end{enumerate}
\end{enumerate}

Виды взаимодействия с пользователями:

\begin{enumerate}
    \item графическая CAPTCHA: выбор изображений, соответствующих запросу;
    \item текстовая CAPTCHA: ввод символов (редко используется);
    \item аудио CAPTCHA: для пользователей с ограниченными возможностями, 
    предлагается прослушать и ввести услышанные символы;
    \item клик CAPTCHA: нажатие на флажок «Я не робот» (для низкорискованных 
    пользователей).
\end{enumerate}

\section{Capy}

Capy CAPTCHA -- это инновационная система CAPTCHA, разработанная с акцентом на 
удобство для пользователей и адаптацию к современным web-средам. Она предлагает 
интерактивные методы проверки, направленные на минимизацию раздражения 
пользователей при сохранении высокого уровня защиты от ботов [6].

Основные особенности Capy CAPTCHA:

\begin{enumerate}
    \item интерактивность:
    \begin{enumerate}
        \item Capy использует методы проверки, которые требуют не просто ввода 
        текста или выбора картинок, а выполнения задач, таких как перемещение 
        объектов;
        \item простые задачи делают процесс проверки менее раздражающим и более 
        интуитивным;
    \end{enumerate}
    \item гибкость настройки:
    \begin{enumerate}
        \item система может быть адаптирована под конкретные нужды сайта, 
        включая выбор сложности задач и дизайн интерфейса.
    \end{enumerate}
    \item доступность:
    \begin{enumerate}
        \item подходит для пользователей с различными потребностями, включая 
        мобильные устройства.
    \end{enumerate}
\end{enumerate}

Виды взаимодействия с пользователями:

\begin{enumerate}
    \item головоломки (Puzzle CAPTCHA): сборка пазла с перемещением недостающих 
    элементов в нужное место;
    \item тесты на логику и распознавание: выбор нужного объекта или 
    логического варианта из предложенных;
    \item текстовая CAPTCHA (редко используется).
\end{enumerate}

Capy CAPTCHA используется на сайтах, где важны как защита от ботов, так и 
положительный пользовательский опыт. Особенно популярна в проектах с высоким 
акцентом на дизайн и пользовательское взаимодействие.

% Глава про разработку всех решателей

\chapter{Методология решения задач CAPTCHA}

\section{Общие подходы к автоматизированному решению CAPTCHA}

Для автоматизации решения CAPTCHA могут применяться различные методы и подходы, 
которые зависят от конкретной реализации CAPTCHA. В рамках данной работы можно 
выделить следующий список таких методов:

\begin{enumerate}
    \item cинтез речи;
    \item сегментация;
    \item классификация;
    \item распознавание последовательности;
    \item может быть продлю...
\end{enumerate}

% TODO: Расписать каждый подход с указанием литературы

\section{Архитектуры нейросетей для различных форматов CAPTCHA}

В данной работе рассмотрены три наиболее популярные и частовстречающиеся 
реализации CAPTCHA, которые применяются для защиты web-ресурсов: аудио CAPTCHA, 
текстовые CAPTCHA и графические CAPTCHA (CAPTCHA с изображениями). Для каждой 
реализации необходим свой подход к решению, разный набор инструментов и библиотек.

Далее для каждой из реализаций будцт рассмотрены различные архитектуры нейронных 
сетей, которые могут быть использованы для автоматизации решения CAPTCHA.

\textbf{Архитектуры нейронных сетей для аудио CAPTCHA}

Для аудио CAPTCHA доступны следующие архитектуры нейронных сетей и инструменты:

\begin{enumerate}
    \item Открытые API для работы с языковыми моделями от Google, Microsoft и 
    других;
    \item Написать еще несколько вариантов...
\end{enumerate}

% TODO: Более подпробно рассмотреть каждый из вариантов с привлечением литературы и так далее

\textbf{Архитектуры нейронных сетей для текстовых CAPTCHA}

Для задачи решения текстовых CAPTCHA могут быть использованы различные модели 
нейронных сетей, которые поддерживают обработку последовательностей различной 
длины. Среди таких архитектур и инструментов можно выделить следующие:

\begin{enumerate}
    \item Tesseract OCR;
    \item CRNN + CTC;
    \item Sequence-to-Sequence.
\end{enumerate}

% TODO: Добавить больше теоретической информации с привлечением литературы

\textbf{Архитектуры нейронных сетей для графических CAPTCHA}

При решении графических CAPTCHA важными являются возможности модели по детекции
и сегментации объектов, поскольку данные CAPTCHA могут требовать как обычного 
поиска объекта, так и выбора клеток, в которых содержится объект. Для решения 
данныхзадач могут применяться следующие инструменты и архитектуры нейронных сетей:

\begin{enumerate}
    \item YOLO;
    \item DETR;
    \item Faster R-CNN.
\end{enumerate}

% TODO: Добавить больше теоретической информации с привлечением литературы

\section{Подготовка и аннотация датасетов}

\textbf{Подготовка датасета для тектcовых CAPTCHA}

Поскольку в открытом доступе отсутствует достаточное количество данных для 
формирования сбалансированного датасета, было принято решение о генерации 
синтетических изображений с использованием специализированных библиотек. В 
качестве основного инструмента выбрана библиотека captcha на языке Python, 
обладающая необходимым функционалом для создания изображений CAPTCHA с 
заданными параметрами. Данная библиотека поддерживает генерацию изображений с 
пользовательскими шрифтами и различными эффектами искажений, что исключает 
необходимость привлечения дополнительных инструментов.

Исходный код генератора синтетических CAPTCHA представлен в приложении~
\ref{code:gen-dataset}.

После создания изображений все они прошли этапы предобработки, направленные на 
улучшение качества данных и повышение эффективности обучения модели. 
Предобработка включала следующие этапы:

\begin{enumerate}
    \item преобразование изображений в градации серого для уменьшения количества 
    каналов и снижения вычислительной нагрузки;
    \item бинаризация изображений с целью получения контрастного представления 
    символов (белый текст на черном фоне);
    \item удаление шумов и фона с использованием морфологических операций, в 
    частности, дилатации.
\end{enumerate}

Исходный код обработчика изображений представлен в приложении~
\ref{code:preprocessing}.

Примеры сгенерированных и предобработанных CAPTCHA приведены на рисунке ниже:

\begin{figure}[H]
    \centering
    \begin{minipage}[h]{0.45\linewidth}
        \center{\includegraphics[width=1\linewidth]{imgs/textcaptcha/YKQ9.png}} 
        \\ а)
    \end{minipage}
    \begin{minipage}[h]{0.45\linewidth}
        \center{\includegraphics[width=1\linewidth]{imgs/textcaptcha/out.png}} 
        \\ б)
    \end{minipage}
    \caption{Изображения CAPTCHA: а) -- сгенерированное изображение, б) -- 
    результат обработки.}
    \label{fig:example-captcha}
\end{figure}

\vspace{-0.7cm}

\textbf{Подготовка датасета для CAPTCHA с изображениями}

Большинство предобученных моделей компьютерного зрения, таких как YOLOv8, обучены 
на датасете COCO~\cite{COCO}, содержащем изображения высокого качества с чёткими 
контурами и однозначной аннотацией объектов. Однако CAPTCHA с изображениями имеют 
принципиально иные характеристики: они могут включать в себя размытие, наложенные 
артефакты, искажения, шумы, повторяющиеся элементы и искусственно пониженное 
разрешение. Всё это снижает эффективность использования стандартных датасетов и 
моделей, не адаптированных под такие условия.

Для обеспечения высокой точности в задаче автоматического решения CAPTCHA 
необходимо подготовить собственный набор данных, приближённый к реальным условиям 
использования. Наиболее эффективным методом является автоматизированный парсинг 
изображений CAPTCHA, представленных на веб-сайтах, использующих визуальные 
CAPTCHA-решения, такие как Google reCAPTCHA v2.

Использование реальных CAPTCHA, собранных в автоматическом режиме, имеет ряд 
преимуществ по сравнению с синтетической генерацией данных:

\begin{enumerate}
    \item изображения содержат разнообразные сцены, освещение, углы обзора и 
    уровни шума, что положительно влияет на способность модели к обобщению;
    \item присутствует большое количество уникальных объектов на фоне, в том 
    числе в частично перекрытых и смазанных вариантах;
    \item отсутствует необходимость в ручной генерации изображений и создании 
    дополнительных искажений для повышения реалистичности;
    \item возможно извлекать текстовые инструкции к CAPTCHA, что позволяет 
    соотносить каждое изображение с требуемым классом.
\end{enumerate}

Для парсинга CAPTCHA был реализован автоматизированный сценарий взаимодействия с 
браузером с использованием библиотеки Selenium~\cite{Selenium}. Данный подход 
позволяет воспроизвести действия пользователя при работе с CAPTCHA, обходя при 
этом ручной ввод. Для обеспечения стабильной работы и масштабируемости процесса 
применялась браузерная автоматизация через WebDriver (в частности, ChromeDriver).

Функциональность парсера включает следующие ключевые этапы:

\begin{enumerate}
    \item поиск iframe-элемента, содержащего чекбокс <<Я не робот>>, и эмуляция 
    клика по нему для инициирования визуальной CAPTCHA;
    \item ожидание загрузки CAPTCHA и извлечение изображения с заданием (включая 
    его URL или пиксельный снимок);
    \item извлечение информации о структуре сетки (количество строк и столбцов), 
    на которую разбито изображение CAPTCHA;
    \item получение текста задания, содержащего имя объекта (например, <<выберите 
    все изображения с мотоциклами>>), для последующего использования в аннотации 
    данных.
\end{enumerate}

Типичная CAPTCHA представляет собой изображение, разделённое на сетку из 3×3 или 
4×4 ячеек, каждая из которых может содержать фрагмент сцены. При этом 
пользователю предлагается выбрать ячейки, в которых присутствует объект заданного 
класса. Процесс парсинга может быть представлена блок-схемой на рис.~
\ref{fig:captcha-flow}.

\begin{figure}[H]
    \centering
    \includegraphics[width=0.6\textwidth]{
        imgs/imagecaptcha/image_captcha_flow.png
    }
    \caption{Блок-схема процесса парсинга CAPTCHA.}
    \label{fig:captcha-flow}
\end{figure}
\vspace{-0.5cm}

Полученные изображения и метаданные (включая текст задания и параметры сетки) 
используются для формирования обучающего датасета, пригодного для дообучения 
модели YOLOv8 в задачах классификации и сегментации объектов.

После получения достаточного количества изображений для составления датасета 
необходимо провести их предварительную обработку и разметку. Это один из 
самыхважных этапов работы, поскольку от качества разметки напрямую зависит 
точность и эффективность последующей работы модели.

Для корректной работы модели YOLO требуется создать иерархическую структуру 
папок, в которой изображения и соответствующие метки будут разделены на 
тренировочную и валидационную выборки. Стандартная структура включает следующие 
директории:

\begin{enumerate}
    \item Директория train -- содержит тренировочную выборку:
    \begin{enumerate}
        \item images -- изображения;
        \item labels -- метки к изображениям.
    \end{enumerate}
    \item Директория val -- содержит валидационную выборку:
    \begin{enumerate}
        \item images -- изображения;
        \item labels -- метки к изображениям.
    \end{enumerate}
\end{enumerate}

Набор классов, пути к выборкам и параметры конфигурации задаются в YAML-файле, 
который передается при обучении модели. Содержимое такого файла для данной 
модели:

\begin{code}
    \captionof{listing}{
        \label{code:train-captcha}Параметры конфигурации для обучения модели
    }
    \vspace{-0.75cm}
    {\small
        \inputminted[mathescape,linenos,frame=lines,breaklines]{yaml}{code/imagecaptcha/train_captcha.yaml}
    }
\end{code}
\vspace{-0.4cm}

Для создания меток используется инструмент CVAT (Computer Vision Annotation Tool) 
-- многофункциональное веб-приложение с поддержкой аннотации объектов с помощью 
полигонов, прямоугольников и других форм. CVAT позволяет экспортировать разметку 
напрямую в формат, совместимый с YOLO~\cite{CVAT}.

Поскольку CAPTCHA-изображения часто содержат объекты с нечёткими контурами, 
наложением и визуальными искажениями, особенно важно использовать ручную точную 
разметку, а не ограничиваться автоматическими методами. Выделение объектов должно 
проводиться как можно точнее, с учётом геометрии контуров. На рисунке ниже 
представлен пример изображения с размеченными объектами:

\begin{figure}[H]
    \centering
    \includegraphics[width=0.9\linewidth]{imgs/imagecaptcha/captcha-poligons.png}
    \caption{Пример разметки изображения с тестовой CAPTCHA.}
    \label{fig:mask-captcha}
\end{figure}
\vspace{-0.5cm}

Кроме того, разметка позволяет учесть сразу несколько объектов разных классов на 
одном изображении, что особенно характерно для CAPTCHA, где в одной сетке могут 
одновременно находиться, например, автомобили и автобусы. Такой подход 
положительно влияет на обобщающую способность модели.

В случае, если количество данных по отдельным классам окажется недостаточным, 
можно дополнительно использовать методы аугментации: вращение, масштабирование, 
искажение цвета и контраста. Однако при хорошо организованном парсинге и разметке 
зачастую удается обойтись без аугментации.

% Глава (возможно с какими-то исследованиями)
\chapter*{ЗАКЛЮЧЕНИЕ}
\addcontentsline{toc}{chapter}{ЗАКЛЮЧЕНИЕ}

В ходе выполнения данной работы была рассмотрена проблема автоматизации 
распознавания CAPTCHA различных форматов с применением современных нейросетевых и 
программных инструментов. Актуальность исследования обусловлена постоянным 
усложнением CAPTCHA-систем и параллельным развитием технологий машинного 
обучения, позволяющих преодолевать подобные механизмы защиты.

В рамках исследования поставленная цель: разработка и анализ комплексного подхода 
к автоматизации решения CAPTCHA в различных форматах с использованием современных 
нейросетевых инструментов и API для распознавания -- была достигнута.

По результатам работы были решены следующие задачи:

\begin{enumerate}
    \item проведён обзор форматов CAPTCHA и существующих методов защиты от 
    автоматических атак;
    \item реализована система для распознавания текстовых CAPTCHA на основе 
    нейросетевого подхода, обеспечивающего устойчивость к искажениям и фоновому 
    шуму;
    \item создано решение для графических CAPTCHA с использованием модели YOLO, 
    адаптированной для распознавания объектов на изображениях;
    \item реализован подход к решению CAPTCHA в аудиоформате с использованием 
    облачного API распознавания речи, обладающего высокой точностью в условиях 
    фоновых шумов;
    \item проведено тестирование всех компонентов системы в условиях, 
    приближенных к реальным, с подтверждением их корректной и стабильной работы.
\end{enumerate}

Полученные результаты демонстрируют возможность эффективного распознавания 
различных типов CAPTCHA при помощи специализированных моделей и сервисов. 
Решения, основанные на применении нейросетей и облачных технологий, показали 
достаточную точность и адаптивность к искажениям в форматах защиты.

Перспективы дальнейших исследований включают:

\begin{enumerate}
    \item расширение набора поддерживаемых типов CAPTCHA, включая более сложные 
    динамические и мультимодальные варианты;
    \item оптимизацию времени обработки и точности распознавания;
    \item исследование механизмов защиты CAPTCHA, устойчивых к современным 
    методам автоматического анализа.
\end{enumerate}

Таким образом, предложенный подход демонстрирует практическую применимость 
современных инструментов машинного обучения и компьютерного зрения в задачах 
анализа и распознавания CAPTCHA, а также может служить основой для дальнейших 
разработок в области тестирования надёжности и устойчивости защитных механизмов 
на web-ресурсах.


\newpage
\addcontentsline{toc}{chapter}{СПИСОК ИСПОЛЬЗОВАННОЙ ЛИТЕРАТУРЫ}
\printbibliography[title={СПИСОК ИСПОЛЬЗОВАННОЙ ЛИТЕРАТУРЫ}]

\appendix
\newpage
\chapter*{\begin{flushright}\raggedleft\label{appendix1}Приложение 1\end{flushright}}
\phantomsection\addcontentsline{toc}{chapter}{ПРИЛОЖЕНИЕ 1}
\vspace{-1.75cm}

\begin{center}
\label{code:appendix}Текст программ для автоматизации решения аудио CAPTCHA
\end{center}

\begin{code}
    \captionof{listing}{\label{code:audiocaptcha}Исходный код расшифровки аудио CAPTCHA}
    \vspace{-1cm}
    \inputminted{python}{code/audiocaptcha/audiocaptcha.py}
\end{code}

\begin{code}
    \captionof{listing}{\label{code:audiocaptcha-solve}Исходный код автоматизированного решения Audio CAPTCHA}
    \vspace{-1cm}
    \inputminted{python}{code/audiocaptcha/audiocaptcha_solve.py}
\end{code}

\appendix
\newpage
\chapter*{\begin{flushright}\raggedleft\label{appendix2}Приложение 2\end{flushright}}
\phantomsection\addcontentsline{toc}{chapter}{ПРИЛОЖЕНИЕ 2}
\vspace{-1.75cm}

\begin{center}
\label{code:appendix2}Текст программ для автоматизации решения текстовых CAPTCHA
\end{center}

\begin{code}
    \captionof{listing}{\label{code:gen-dataset}Исходный код генератора синтетических текстовых CAPTCHA}
    \vspace{-1cm}
    \inputminted{python}{code/textcaptcha/gen-dataset.py}
\end{code}

\begin{code}
    \captionof{listing}{\label{code:preprocessing}Исходный код для предобработки изображений датасета с текстовыми CAPTCHA}
    \vspace{-1cm}
    \inputminted{python}{code/textcaptcha/preprocessing.py}
\end{code}

\begin{code}
    \captionof{listing}{\label{code:tf-dataset}Исходный код для создания датасета текстовых CAPTCHA в формате тензоров}
    \vspace{-1cm}
    \inputminted{python}{code/textcaptcha/tf-dataset.py}
\end{code}

\begin{code}
    \captionof{listing}{\label{code:crnn}Исходный код CRNN модели}
    \vspace{-1cm}
    \inputminted{python}{code/textcaptcha/crnn.py}
\end{code}

\begin{code}
    \captionof{listing}{\label{code:seq2seq}Исходный код Seq2Seq модели}
    \vspace{-1cm}
    \inputminted{python}{code/textcaptcha/seq-to-seq.py}
\end{code}

\begin{code}
    \captionof{listing}{\label{code:test-model}Исходный код тестирования Seq2Seq модели}
    \vspace{-1cm}
    \inputminted{python}{code/textcaptcha/test.py}
\end{code}

\appendix
\newpage
\chapter*{\begin{flushright}\raggedleft\label{appendix3}Приложение 3\end{flushright}}
\phantomsection\addcontentsline{toc}{chapter}{ПРИЛОЖЕНИЕ 3}
\vspace{-1.75cm}

\begin{center}
\label{code:appendix3}Текст программ для автоматизации решения графических CAPTCHA
\end{center}

\begin{code}
    \captionof{listing}{\label{code:get-captcha}Исходный код получения графических CAPTCHA с целевого сайта}
    \vspace{-1cm}
    \inputminted{python}{code/imagecaptcha/get_captcha.py}
\end{code}

\begin{code}
    \captionof{listing}{\label{code:recognize}Исходный код дообучения модели на датасете с графическими CAPTCHA}
    \vspace{-1cm}
    \inputminted{python}{code/imagecaptcha/recognize_objects.py}
\end{code}

\begin{code}
    \captionof{listing}{\label{code:solve-captcha}Исходный код автоматизированного решения графических CAPTCHA}
    \vspace{-1cm}
    \inputminted{python}{code/imagecaptcha/solve_captcha.py}
\end{code}

\makelastpage
\end{document}
